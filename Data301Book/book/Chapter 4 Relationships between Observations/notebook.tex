
% Default to the notebook output style

    


% Inherit from the specified cell style.




    
\documentclass[11pt]{article}

    
    
    \usepackage[T1]{fontenc}
    % Nicer default font (+ math font) than Computer Modern for most use cases
    \usepackage{mathpazo}

    % Basic figure setup, for now with no caption control since it's done
    % automatically by Pandoc (which extracts ![](path) syntax from Markdown).
    \usepackage{graphicx}
    % We will generate all images so they have a width \maxwidth. This means
    % that they will get their normal width if they fit onto the page, but
    % are scaled down if they would overflow the margins.
    \makeatletter
    \def\maxwidth{\ifdim\Gin@nat@width>\linewidth\linewidth
    \else\Gin@nat@width\fi}
    \makeatother
    \let\Oldincludegraphics\includegraphics
    % Set max figure width to be 80% of text width, for now hardcoded.
    \renewcommand{\includegraphics}[1]{\Oldincludegraphics[width=.8\maxwidth]{#1}}
    % Ensure that by default, figures have no caption (until we provide a
    % proper Figure object with a Caption API and a way to capture that
    % in the conversion process - todo).
    \usepackage{caption}
    \DeclareCaptionLabelFormat{nolabel}{}
    \captionsetup{labelformat=nolabel}

    \usepackage{adjustbox} % Used to constrain images to a maximum size 
    \usepackage{xcolor} % Allow colors to be defined
    \usepackage{enumerate} % Needed for markdown enumerations to work
    \usepackage{geometry} % Used to adjust the document margins
    \usepackage{amsmath} % Equations
    \usepackage{amssymb} % Equations
    \usepackage{textcomp} % defines textquotesingle
    % Hack from http://tex.stackexchange.com/a/47451/13684:
    \AtBeginDocument{%
        \def\PYZsq{\textquotesingle}% Upright quotes in Pygmentized code
    }
    \usepackage{upquote} % Upright quotes for verbatim code
    \usepackage{eurosym} % defines \euro
    \usepackage[mathletters]{ucs} % Extended unicode (utf-8) support
    \usepackage[utf8x]{inputenc} % Allow utf-8 characters in the tex document
    \usepackage{fancyvrb} % verbatim replacement that allows latex
    \usepackage{grffile} % extends the file name processing of package graphics 
                         % to support a larger range 
    % The hyperref package gives us a pdf with properly built
    % internal navigation ('pdf bookmarks' for the table of contents,
    % internal cross-reference links, web links for URLs, etc.)
    \usepackage{hyperref}
    \usepackage{longtable} % longtable support required by pandoc >1.10
    \usepackage{booktabs}  % table support for pandoc > 1.12.2
    \usepackage[inline]{enumitem} % IRkernel/repr support (it uses the enumerate* environment)
    \usepackage[normalem]{ulem} % ulem is needed to support strikethroughs (\sout)
                                % normalem makes italics be italics, not underlines
    

    
    
    % Colors for the hyperref package
    \definecolor{urlcolor}{rgb}{0,.145,.698}
    \definecolor{linkcolor}{rgb}{.71,0.21,0.01}
    \definecolor{citecolor}{rgb}{.12,.54,.11}

    % ANSI colors
    \definecolor{ansi-black}{HTML}{3E424D}
    \definecolor{ansi-black-intense}{HTML}{282C36}
    \definecolor{ansi-red}{HTML}{E75C58}
    \definecolor{ansi-red-intense}{HTML}{B22B31}
    \definecolor{ansi-green}{HTML}{00A250}
    \definecolor{ansi-green-intense}{HTML}{007427}
    \definecolor{ansi-yellow}{HTML}{DDB62B}
    \definecolor{ansi-yellow-intense}{HTML}{B27D12}
    \definecolor{ansi-blue}{HTML}{208FFB}
    \definecolor{ansi-blue-intense}{HTML}{0065CA}
    \definecolor{ansi-magenta}{HTML}{D160C4}
    \definecolor{ansi-magenta-intense}{HTML}{A03196}
    \definecolor{ansi-cyan}{HTML}{60C6C8}
    \definecolor{ansi-cyan-intense}{HTML}{258F8F}
    \definecolor{ansi-white}{HTML}{C5C1B4}
    \definecolor{ansi-white-intense}{HTML}{A1A6B2}

    % commands and environments needed by pandoc snippets
    % extracted from the output of `pandoc -s`
    \providecommand{\tightlist}{%
      \setlength{\itemsep}{0pt}\setlength{\parskip}{0pt}}
    \DefineVerbatimEnvironment{Highlighting}{Verbatim}{commandchars=\\\{\}}
    % Add ',fontsize=\small' for more characters per line
    \newenvironment{Shaded}{}{}
    \newcommand{\KeywordTok}[1]{\textcolor[rgb]{0.00,0.44,0.13}{\textbf{{#1}}}}
    \newcommand{\DataTypeTok}[1]{\textcolor[rgb]{0.56,0.13,0.00}{{#1}}}
    \newcommand{\DecValTok}[1]{\textcolor[rgb]{0.25,0.63,0.44}{{#1}}}
    \newcommand{\BaseNTok}[1]{\textcolor[rgb]{0.25,0.63,0.44}{{#1}}}
    \newcommand{\FloatTok}[1]{\textcolor[rgb]{0.25,0.63,0.44}{{#1}}}
    \newcommand{\CharTok}[1]{\textcolor[rgb]{0.25,0.44,0.63}{{#1}}}
    \newcommand{\StringTok}[1]{\textcolor[rgb]{0.25,0.44,0.63}{{#1}}}
    \newcommand{\CommentTok}[1]{\textcolor[rgb]{0.38,0.63,0.69}{\textit{{#1}}}}
    \newcommand{\OtherTok}[1]{\textcolor[rgb]{0.00,0.44,0.13}{{#1}}}
    \newcommand{\AlertTok}[1]{\textcolor[rgb]{1.00,0.00,0.00}{\textbf{{#1}}}}
    \newcommand{\FunctionTok}[1]{\textcolor[rgb]{0.02,0.16,0.49}{{#1}}}
    \newcommand{\RegionMarkerTok}[1]{{#1}}
    \newcommand{\ErrorTok}[1]{\textcolor[rgb]{1.00,0.00,0.00}{\textbf{{#1}}}}
    \newcommand{\NormalTok}[1]{{#1}}
    
    % Additional commands for more recent versions of Pandoc
    \newcommand{\ConstantTok}[1]{\textcolor[rgb]{0.53,0.00,0.00}{{#1}}}
    \newcommand{\SpecialCharTok}[1]{\textcolor[rgb]{0.25,0.44,0.63}{{#1}}}
    \newcommand{\VerbatimStringTok}[1]{\textcolor[rgb]{0.25,0.44,0.63}{{#1}}}
    \newcommand{\SpecialStringTok}[1]{\textcolor[rgb]{0.73,0.40,0.53}{{#1}}}
    \newcommand{\ImportTok}[1]{{#1}}
    \newcommand{\DocumentationTok}[1]{\textcolor[rgb]{0.73,0.13,0.13}{\textit{{#1}}}}
    \newcommand{\AnnotationTok}[1]{\textcolor[rgb]{0.38,0.63,0.69}{\textbf{\textit{{#1}}}}}
    \newcommand{\CommentVarTok}[1]{\textcolor[rgb]{0.38,0.63,0.69}{\textbf{\textit{{#1}}}}}
    \newcommand{\VariableTok}[1]{\textcolor[rgb]{0.10,0.09,0.49}{{#1}}}
    \newcommand{\ControlFlowTok}[1]{\textcolor[rgb]{0.00,0.44,0.13}{\textbf{{#1}}}}
    \newcommand{\OperatorTok}[1]{\textcolor[rgb]{0.40,0.40,0.40}{{#1}}}
    \newcommand{\BuiltInTok}[1]{{#1}}
    \newcommand{\ExtensionTok}[1]{{#1}}
    \newcommand{\PreprocessorTok}[1]{\textcolor[rgb]{0.74,0.48,0.00}{{#1}}}
    \newcommand{\AttributeTok}[1]{\textcolor[rgb]{0.49,0.56,0.16}{{#1}}}
    \newcommand{\InformationTok}[1]{\textcolor[rgb]{0.38,0.63,0.69}{\textbf{\textit{{#1}}}}}
    \newcommand{\WarningTok}[1]{\textcolor[rgb]{0.38,0.63,0.69}{\textbf{\textit{{#1}}}}}
    
    
    % Define a nice break command that doesn't care if a line doesn't already
    % exist.
    \def\br{\hspace*{\fill} \\* }
    % Math Jax compatability definitions
    \def\gt{>}
    \def\lt{<}
    % Document parameters
    \title{Exam1}
    
    
    

    % Pygments definitions
    
\makeatletter
\def\PY@reset{\let\PY@it=\relax \let\PY@bf=\relax%
    \let\PY@ul=\relax \let\PY@tc=\relax%
    \let\PY@bc=\relax \let\PY@ff=\relax}
\def\PY@tok#1{\csname PY@tok@#1\endcsname}
\def\PY@toks#1+{\ifx\relax#1\empty\else%
    \PY@tok{#1}\expandafter\PY@toks\fi}
\def\PY@do#1{\PY@bc{\PY@tc{\PY@ul{%
    \PY@it{\PY@bf{\PY@ff{#1}}}}}}}
\def\PY#1#2{\PY@reset\PY@toks#1+\relax+\PY@do{#2}}

\expandafter\def\csname PY@tok@w\endcsname{\def\PY@tc##1{\textcolor[rgb]{0.73,0.73,0.73}{##1}}}
\expandafter\def\csname PY@tok@c\endcsname{\let\PY@it=\textit\def\PY@tc##1{\textcolor[rgb]{0.25,0.50,0.50}{##1}}}
\expandafter\def\csname PY@tok@cp\endcsname{\def\PY@tc##1{\textcolor[rgb]{0.74,0.48,0.00}{##1}}}
\expandafter\def\csname PY@tok@k\endcsname{\let\PY@bf=\textbf\def\PY@tc##1{\textcolor[rgb]{0.00,0.50,0.00}{##1}}}
\expandafter\def\csname PY@tok@kp\endcsname{\def\PY@tc##1{\textcolor[rgb]{0.00,0.50,0.00}{##1}}}
\expandafter\def\csname PY@tok@kt\endcsname{\def\PY@tc##1{\textcolor[rgb]{0.69,0.00,0.25}{##1}}}
\expandafter\def\csname PY@tok@o\endcsname{\def\PY@tc##1{\textcolor[rgb]{0.40,0.40,0.40}{##1}}}
\expandafter\def\csname PY@tok@ow\endcsname{\let\PY@bf=\textbf\def\PY@tc##1{\textcolor[rgb]{0.67,0.13,1.00}{##1}}}
\expandafter\def\csname PY@tok@nb\endcsname{\def\PY@tc##1{\textcolor[rgb]{0.00,0.50,0.00}{##1}}}
\expandafter\def\csname PY@tok@nf\endcsname{\def\PY@tc##1{\textcolor[rgb]{0.00,0.00,1.00}{##1}}}
\expandafter\def\csname PY@tok@nc\endcsname{\let\PY@bf=\textbf\def\PY@tc##1{\textcolor[rgb]{0.00,0.00,1.00}{##1}}}
\expandafter\def\csname PY@tok@nn\endcsname{\let\PY@bf=\textbf\def\PY@tc##1{\textcolor[rgb]{0.00,0.00,1.00}{##1}}}
\expandafter\def\csname PY@tok@ne\endcsname{\let\PY@bf=\textbf\def\PY@tc##1{\textcolor[rgb]{0.82,0.25,0.23}{##1}}}
\expandafter\def\csname PY@tok@nv\endcsname{\def\PY@tc##1{\textcolor[rgb]{0.10,0.09,0.49}{##1}}}
\expandafter\def\csname PY@tok@no\endcsname{\def\PY@tc##1{\textcolor[rgb]{0.53,0.00,0.00}{##1}}}
\expandafter\def\csname PY@tok@nl\endcsname{\def\PY@tc##1{\textcolor[rgb]{0.63,0.63,0.00}{##1}}}
\expandafter\def\csname PY@tok@ni\endcsname{\let\PY@bf=\textbf\def\PY@tc##1{\textcolor[rgb]{0.60,0.60,0.60}{##1}}}
\expandafter\def\csname PY@tok@na\endcsname{\def\PY@tc##1{\textcolor[rgb]{0.49,0.56,0.16}{##1}}}
\expandafter\def\csname PY@tok@nt\endcsname{\let\PY@bf=\textbf\def\PY@tc##1{\textcolor[rgb]{0.00,0.50,0.00}{##1}}}
\expandafter\def\csname PY@tok@nd\endcsname{\def\PY@tc##1{\textcolor[rgb]{0.67,0.13,1.00}{##1}}}
\expandafter\def\csname PY@tok@s\endcsname{\def\PY@tc##1{\textcolor[rgb]{0.73,0.13,0.13}{##1}}}
\expandafter\def\csname PY@tok@sd\endcsname{\let\PY@it=\textit\def\PY@tc##1{\textcolor[rgb]{0.73,0.13,0.13}{##1}}}
\expandafter\def\csname PY@tok@si\endcsname{\let\PY@bf=\textbf\def\PY@tc##1{\textcolor[rgb]{0.73,0.40,0.53}{##1}}}
\expandafter\def\csname PY@tok@se\endcsname{\let\PY@bf=\textbf\def\PY@tc##1{\textcolor[rgb]{0.73,0.40,0.13}{##1}}}
\expandafter\def\csname PY@tok@sr\endcsname{\def\PY@tc##1{\textcolor[rgb]{0.73,0.40,0.53}{##1}}}
\expandafter\def\csname PY@tok@ss\endcsname{\def\PY@tc##1{\textcolor[rgb]{0.10,0.09,0.49}{##1}}}
\expandafter\def\csname PY@tok@sx\endcsname{\def\PY@tc##1{\textcolor[rgb]{0.00,0.50,0.00}{##1}}}
\expandafter\def\csname PY@tok@m\endcsname{\def\PY@tc##1{\textcolor[rgb]{0.40,0.40,0.40}{##1}}}
\expandafter\def\csname PY@tok@gh\endcsname{\let\PY@bf=\textbf\def\PY@tc##1{\textcolor[rgb]{0.00,0.00,0.50}{##1}}}
\expandafter\def\csname PY@tok@gu\endcsname{\let\PY@bf=\textbf\def\PY@tc##1{\textcolor[rgb]{0.50,0.00,0.50}{##1}}}
\expandafter\def\csname PY@tok@gd\endcsname{\def\PY@tc##1{\textcolor[rgb]{0.63,0.00,0.00}{##1}}}
\expandafter\def\csname PY@tok@gi\endcsname{\def\PY@tc##1{\textcolor[rgb]{0.00,0.63,0.00}{##1}}}
\expandafter\def\csname PY@tok@gr\endcsname{\def\PY@tc##1{\textcolor[rgb]{1.00,0.00,0.00}{##1}}}
\expandafter\def\csname PY@tok@ge\endcsname{\let\PY@it=\textit}
\expandafter\def\csname PY@tok@gs\endcsname{\let\PY@bf=\textbf}
\expandafter\def\csname PY@tok@gp\endcsname{\let\PY@bf=\textbf\def\PY@tc##1{\textcolor[rgb]{0.00,0.00,0.50}{##1}}}
\expandafter\def\csname PY@tok@go\endcsname{\def\PY@tc##1{\textcolor[rgb]{0.53,0.53,0.53}{##1}}}
\expandafter\def\csname PY@tok@gt\endcsname{\def\PY@tc##1{\textcolor[rgb]{0.00,0.27,0.87}{##1}}}
\expandafter\def\csname PY@tok@err\endcsname{\def\PY@bc##1{\setlength{\fboxsep}{0pt}\fcolorbox[rgb]{1.00,0.00,0.00}{1,1,1}{\strut ##1}}}
\expandafter\def\csname PY@tok@kc\endcsname{\let\PY@bf=\textbf\def\PY@tc##1{\textcolor[rgb]{0.00,0.50,0.00}{##1}}}
\expandafter\def\csname PY@tok@kd\endcsname{\let\PY@bf=\textbf\def\PY@tc##1{\textcolor[rgb]{0.00,0.50,0.00}{##1}}}
\expandafter\def\csname PY@tok@kn\endcsname{\let\PY@bf=\textbf\def\PY@tc##1{\textcolor[rgb]{0.00,0.50,0.00}{##1}}}
\expandafter\def\csname PY@tok@kr\endcsname{\let\PY@bf=\textbf\def\PY@tc##1{\textcolor[rgb]{0.00,0.50,0.00}{##1}}}
\expandafter\def\csname PY@tok@bp\endcsname{\def\PY@tc##1{\textcolor[rgb]{0.00,0.50,0.00}{##1}}}
\expandafter\def\csname PY@tok@fm\endcsname{\def\PY@tc##1{\textcolor[rgb]{0.00,0.00,1.00}{##1}}}
\expandafter\def\csname PY@tok@vc\endcsname{\def\PY@tc##1{\textcolor[rgb]{0.10,0.09,0.49}{##1}}}
\expandafter\def\csname PY@tok@vg\endcsname{\def\PY@tc##1{\textcolor[rgb]{0.10,0.09,0.49}{##1}}}
\expandafter\def\csname PY@tok@vi\endcsname{\def\PY@tc##1{\textcolor[rgb]{0.10,0.09,0.49}{##1}}}
\expandafter\def\csname PY@tok@vm\endcsname{\def\PY@tc##1{\textcolor[rgb]{0.10,0.09,0.49}{##1}}}
\expandafter\def\csname PY@tok@sa\endcsname{\def\PY@tc##1{\textcolor[rgb]{0.73,0.13,0.13}{##1}}}
\expandafter\def\csname PY@tok@sb\endcsname{\def\PY@tc##1{\textcolor[rgb]{0.73,0.13,0.13}{##1}}}
\expandafter\def\csname PY@tok@sc\endcsname{\def\PY@tc##1{\textcolor[rgb]{0.73,0.13,0.13}{##1}}}
\expandafter\def\csname PY@tok@dl\endcsname{\def\PY@tc##1{\textcolor[rgb]{0.73,0.13,0.13}{##1}}}
\expandafter\def\csname PY@tok@s2\endcsname{\def\PY@tc##1{\textcolor[rgb]{0.73,0.13,0.13}{##1}}}
\expandafter\def\csname PY@tok@sh\endcsname{\def\PY@tc##1{\textcolor[rgb]{0.73,0.13,0.13}{##1}}}
\expandafter\def\csname PY@tok@s1\endcsname{\def\PY@tc##1{\textcolor[rgb]{0.73,0.13,0.13}{##1}}}
\expandafter\def\csname PY@tok@mb\endcsname{\def\PY@tc##1{\textcolor[rgb]{0.40,0.40,0.40}{##1}}}
\expandafter\def\csname PY@tok@mf\endcsname{\def\PY@tc##1{\textcolor[rgb]{0.40,0.40,0.40}{##1}}}
\expandafter\def\csname PY@tok@mh\endcsname{\def\PY@tc##1{\textcolor[rgb]{0.40,0.40,0.40}{##1}}}
\expandafter\def\csname PY@tok@mi\endcsname{\def\PY@tc##1{\textcolor[rgb]{0.40,0.40,0.40}{##1}}}
\expandafter\def\csname PY@tok@il\endcsname{\def\PY@tc##1{\textcolor[rgb]{0.40,0.40,0.40}{##1}}}
\expandafter\def\csname PY@tok@mo\endcsname{\def\PY@tc##1{\textcolor[rgb]{0.40,0.40,0.40}{##1}}}
\expandafter\def\csname PY@tok@ch\endcsname{\let\PY@it=\textit\def\PY@tc##1{\textcolor[rgb]{0.25,0.50,0.50}{##1}}}
\expandafter\def\csname PY@tok@cm\endcsname{\let\PY@it=\textit\def\PY@tc##1{\textcolor[rgb]{0.25,0.50,0.50}{##1}}}
\expandafter\def\csname PY@tok@cpf\endcsname{\let\PY@it=\textit\def\PY@tc##1{\textcolor[rgb]{0.25,0.50,0.50}{##1}}}
\expandafter\def\csname PY@tok@c1\endcsname{\let\PY@it=\textit\def\PY@tc##1{\textcolor[rgb]{0.25,0.50,0.50}{##1}}}
\expandafter\def\csname PY@tok@cs\endcsname{\let\PY@it=\textit\def\PY@tc##1{\textcolor[rgb]{0.25,0.50,0.50}{##1}}}

\def\PYZbs{\char`\\}
\def\PYZus{\char`\_}
\def\PYZob{\char`\{}
\def\PYZcb{\char`\}}
\def\PYZca{\char`\^}
\def\PYZam{\char`\&}
\def\PYZlt{\char`\<}
\def\PYZgt{\char`\>}
\def\PYZsh{\char`\#}
\def\PYZpc{\char`\%}
\def\PYZdl{\char`\$}
\def\PYZhy{\char`\-}
\def\PYZsq{\char`\'}
\def\PYZdq{\char`\"}
\def\PYZti{\char`\~}
% for compatibility with earlier versions
\def\PYZat{@}
\def\PYZlb{[}
\def\PYZrb{]}
\makeatother


    % Exact colors from NB
    \definecolor{incolor}{rgb}{0.0, 0.0, 0.5}
    \definecolor{outcolor}{rgb}{0.545, 0.0, 0.0}



    
    % Prevent overflowing lines due to hard-to-break entities
    \sloppy 
    % Setup hyperref package
    \hypersetup{
      breaklinks=true,  % so long urls are correctly broken across lines
      colorlinks=true,
      urlcolor=urlcolor,
      linkcolor=linkcolor,
      citecolor=citecolor,
      }
    % Slightly bigger margins than the latex defaults
    
    \geometry{verbose,tmargin=1in,bmargin=1in,lmargin=1in,rmargin=1in}
    
    

    \begin{document}
    
    
    \maketitle
    
    

    
    \hypertarget{exam-1}{%
\section{Exam 1}\label{exam-1}}

In this exam, you will work with a data set of 50000 used cars sold in
the United States in 2018. The data set is available here:

\texttt{https://raw.githubusercontent.com/dlsun/data-science-book/master/data/usedcars.csv}

Answer the 8 questions below. The point values are clearly indicated
next to each question. There are 30 possible points.

Some of the questions are deliberately vague. If you are not sure
whether your answer is acceptable, make sure you document your thought
process thoroughly in your explanation.

    \begin{Verbatim}[commandchars=\\\{\}]
{\color{incolor}In [{\color{incolor}1}]:} \PY{c+c1}{\PYZsh{} Import packages.}
        \PY{k+kn}{import} \PY{n+nn}{pandas} \PY{k}{as} \PY{n+nn}{pd}
        \PY{k+kn}{import} \PY{n+nn}{numpy} \PY{k}{as} \PY{n+nn}{np}
        \PY{o}{\PYZpc{}}\PY{k}{matplotlib} inline
        \PY{c+c1}{\PYZsh{} Read in the data set.}
        \PY{n}{cars\PYZus{}df} \PY{o}{=} \PY{n}{pd}\PY{o}{.}\PY{n}{read\PYZus{}csv}\PY{p}{(}\PY{l+s+s2}{\PYZdq{}}\PY{l+s+s2}{https://raw.githubusercontent.com/dlsun/data\PYZhy{}science\PYZhy{}book/master/data/usedcars.csv}\PY{l+s+s2}{\PYZdq{}}\PY{p}{)}
        \PY{n}{cars\PYZus{}df}\PY{p}{[}\PY{p}{[}\PY{l+s+s2}{\PYZdq{}}\PY{l+s+s2}{City}\PY{l+s+s2}{\PYZdq{}}\PY{p}{,} \PY{l+s+s2}{\PYZdq{}}\PY{l+s+s2}{State}\PY{l+s+s2}{\PYZdq{}}\PY{p}{,} \PY{l+s+s2}{\PYZdq{}}\PY{l+s+s2}{Make}\PY{l+s+s2}{\PYZdq{}}\PY{p}{,} \PY{l+s+s2}{\PYZdq{}}\PY{l+s+s2}{Model}\PY{l+s+s2}{\PYZdq{}}\PY{p}{]}\PY{p}{]} \PY{o}{=} \PY{p}{(}
            \PY{n}{cars\PYZus{}df}\PY{p}{[}\PY{p}{[}\PY{l+s+s2}{\PYZdq{}}\PY{l+s+s2}{City}\PY{l+s+s2}{\PYZdq{}}\PY{p}{,} \PY{l+s+s2}{\PYZdq{}}\PY{l+s+s2}{State}\PY{l+s+s2}{\PYZdq{}}\PY{p}{,} \PY{l+s+s2}{\PYZdq{}}\PY{l+s+s2}{Make}\PY{l+s+s2}{\PYZdq{}}\PY{p}{,} \PY{l+s+s2}{\PYZdq{}}\PY{l+s+s2}{Model}\PY{l+s+s2}{\PYZdq{}}\PY{p}{]}\PY{p}{]}
            \PY{o}{.}\PY{n}{apply}\PY{p}{(}\PY{k}{lambda} \PY{n}{s}\PY{p}{:} \PY{n}{s}\PY{o}{.}\PY{n}{str}\PY{o}{.}\PY{n}{strip}\PY{p}{(}\PY{p}{)}\PY{o}{.}\PY{n}{str}\PY{o}{.}\PY{n}{upper}\PY{p}{(}\PY{p}{)}\PY{p}{)}\PY{p}{)}
        \PY{n}{cars\PYZus{}df}\PY{o}{.}\PY{n}{head}\PY{p}{(}\PY{p}{)}
\end{Verbatim}


\begin{Verbatim}[commandchars=\\\{\}]
{\color{outcolor}Out[{\color{outcolor}1}]:}    Price  Year  Mileage           City State                Vin  \textbackslash{}
        0  27947  2010    84925    SAN ANTONIO    TX  JTJJM7FX5A5009042   
        1  16850  2010    76863     RICHARDSON    TX  WDDHF5GB9AA106034   
        2  34974  2017     9978  BUFFALO GROVE    IL  1FM5K8F82HGC23716   
        3  28997  2014    40480   WILLIAMSBURG    VA  1C4RJFBG9EC506630   
        4  12499  2015    21849  DOWNERS GROVE    IL  1FADP3K24FL209932   
        
                    Make           Model  
        0          LEXUS              GX  
        1  MERCEDES-BENZ      E-CLASS4DR  
        2           FORD     EXPLORER4WD  
        3           JEEP           GRAND  
        4           FORD  FOCUSHATCHBACK  
\end{Verbatim}
            
    \hypertarget{question-1-3-points}{%
\section{Question 1 (3 points)}\label{question-1-3-points}}

How is the mileage on a car related to its price? Make a visualization
and report a summary statistic. What general trend do you notice?

    \begin{Verbatim}[commandchars=\\\{\}]
{\color{incolor}In [{\color{incolor}16}]:} \PY{n}{cars\PYZus{}df}\PY{o}{.}\PY{n}{plot}\PY{o}{.}\PY{n}{scatter}\PY{p}{(}\PY{n}{x}\PY{o}{=}\PY{l+s+s2}{\PYZdq{}}\PY{l+s+s2}{Mileage}\PY{l+s+s2}{\PYZdq{}}\PY{p}{,} \PY{n}{y}\PY{o}{=}\PY{l+s+s2}{\PYZdq{}}\PY{l+s+s2}{Price}\PY{l+s+s2}{\PYZdq{}}\PY{p}{,} \PY{n}{title}\PY{o}{=}\PY{l+s+s2}{\PYZdq{}}\PY{l+s+s2}{Price by Mileage}\PY{l+s+s2}{\PYZdq{}}\PY{p}{)}
\end{Verbatim}


\begin{Verbatim}[commandchars=\\\{\}]
{\color{outcolor}Out[{\color{outcolor}16}]:} <matplotlib.axes.\_subplots.AxesSubplot at 0x7f366c4f8550>
\end{Verbatim}
            
    \begin{center}
    \adjustimage{max size={0.9\linewidth}{0.9\paperheight}}{output_3_1.png}
    \end{center}
    { \hspace*{\fill} \\}
    
    \begin{Verbatim}[commandchars=\\\{\}]
{\color{incolor}In [{\color{incolor}3}]:} \PY{n}{cars\PYZus{}df}\PY{p}{[}\PY{l+s+s2}{\PYZdq{}}\PY{l+s+s2}{Mileage}\PY{l+s+s2}{\PYZdq{}}\PY{p}{]}\PY{o}{.}\PY{n}{corr}\PY{p}{(}\PY{n}{cars\PYZus{}df}\PY{p}{[}\PY{l+s+s2}{\PYZdq{}}\PY{l+s+s2}{Price}\PY{l+s+s2}{\PYZdq{}}\PY{p}{]}\PY{p}{)}
\end{Verbatim}


\begin{Verbatim}[commandchars=\\\{\}]
{\color{outcolor}Out[{\color{outcolor}3}]:} -0.42496508318289578
\end{Verbatim}
            
    Mileage and price have a moderate negative relationship. That is, as the
mileage on the car goes up, its price goes down.

    \hypertarget{question-2-3-points}{%
\section{Question 2 (3 points)}\label{question-2-3-points}}

Make a visualization that shows the average price of a car by year. What
general trend do you notice?

    \begin{Verbatim}[commandchars=\\\{\}]
{\color{incolor}In [{\color{incolor}17}]:} \PY{p}{(}\PY{n}{cars\PYZus{}df}\PY{o}{.}\PY{n}{groupby}\PY{p}{(}\PY{l+s+s2}{\PYZdq{}}\PY{l+s+s2}{Year}\PY{l+s+s2}{\PYZdq{}}\PY{p}{)}\PY{o}{.}\PY{n}{Price}\PY{o}{.}\PY{n}{mean}\PY{p}{(}\PY{p}{)}\PY{o}{.}\PY{n}{plot}
             \PY{o}{.}\PY{n}{barh}\PY{p}{(}\PY{n}{figsize}\PY{o}{=}\PY{p}{(}\PY{l+m+mi}{15}\PY{p}{,}\PY{l+m+mi}{15}\PY{p}{)}\PY{p}{,} \PY{n}{title}\PY{o}{=}\PY{l+s+s2}{\PYZdq{}}\PY{l+s+s2}{Avg Price by Year}\PY{l+s+s2}{\PYZdq{}}\PY{p}{)}\PY{p}{)}
\end{Verbatim}


\begin{Verbatim}[commandchars=\\\{\}]
{\color{outcolor}Out[{\color{outcolor}17}]:} <matplotlib.axes.\_subplots.AxesSubplot at 0x7f366c4b2748>
\end{Verbatim}
            
    \begin{center}
    \adjustimage{max size={0.9\linewidth}{0.9\paperheight}}{output_7_1.png}
    \end{center}
    { \hspace*{\fill} \\}
    
    We can see from the bar graph that newer cars in the data set appear to
be more expensive than the older cars in the data set! With that, we can
say that the newer a car is, the higher it may be valued (in terms of
price).

    \hypertarget{question-3-4-points}{%
\section{Question 3 (4 points)}\label{question-3-4-points}}

Restrict to top 10 makes (i.e., the 10 makes that appeared the most
times) in this data set. Make a graphic that shows how the average price
of each make of car changed by year. Your graphic should make it just as
easy to compare the different makes as the different years. Explain what
you see.

    \begin{Verbatim}[commandchars=\\\{\}]
{\color{incolor}In [{\color{incolor}18}]:} \PY{n}{top\PYZus{}10\PYZus{}makes} \PY{o}{=} \PY{n}{cars\PYZus{}df}\PY{p}{[}\PY{n}{cars\PYZus{}df}\PY{o}{.}\PY{n}{Make}\PY{o}{.}\PY{n}{isin}\PY{p}{(}
             \PY{n}{cars\PYZus{}df}\PY{p}{[}\PY{l+s+s2}{\PYZdq{}}\PY{l+s+s2}{Make}\PY{l+s+s2}{\PYZdq{}}\PY{p}{]}\PY{o}{.}\PY{n}{value\PYZus{}counts}\PY{p}{(}\PY{p}{)}\PY{o}{.}\PY{n}{index}\PY{p}{[}\PY{l+m+mi}{1}\PY{p}{:}\PY{l+m+mi}{11}\PY{p}{]}\PY{p}{)}\PY{p}{]}
          
         \PY{p}{(}\PY{n}{top\PYZus{}10\PYZus{}makes}\PY{o}{.}\PY{n}{pivot\PYZus{}table}\PY{p}{(}
             \PY{n}{index}\PY{o}{=}\PY{l+s+s2}{\PYZdq{}}\PY{l+s+s2}{Make}\PY{l+s+s2}{\PYZdq{}}\PY{p}{,} \PY{n}{columns}\PY{o}{=}\PY{l+s+s2}{\PYZdq{}}\PY{l+s+s2}{Year}\PY{l+s+s2}{\PYZdq{}}\PY{p}{,}
             \PY{n}{values}\PY{o}{=}\PY{l+s+s2}{\PYZdq{}}\PY{l+s+s2}{Price}\PY{l+s+s2}{\PYZdq{}}\PY{p}{,} \PY{n}{aggfunc}\PY{o}{=}\PY{n}{np}\PY{o}{.}\PY{n}{mean}\PY{p}{)}
              \PY{o}{.}\PY{n}{plot}\PY{o}{.}\PY{n}{bar}\PY{p}{(}\PY{n}{figsize}\PY{o}{=}\PY{p}{(}\PY{l+m+mi}{20}\PY{p}{,}\PY{l+m+mi}{10}\PY{p}{)}\PY{p}{,} \PY{n}{title}\PY{o}{=}\PY{l+s+s2}{\PYZdq{}}\PY{l+s+s2}{Avg Price by Make and Year}\PY{l+s+s2}{\PYZdq{}}\PY{p}{)}\PY{p}{)}
\end{Verbatim}


\begin{Verbatim}[commandchars=\\\{\}]
{\color{outcolor}Out[{\color{outcolor}18}]:} <matplotlib.axes.\_subplots.AxesSubplot at 0x7f3672d93f60>
\end{Verbatim}
            
    \begin{center}
    \adjustimage{max size={0.9\linewidth}{0.9\paperheight}}{output_10_1.png}
    \end{center}
    { \hspace*{\fill} \\}
    
    From the graphic, we can see that all car makes did in fact increase in
price as the year increased. We can also see that BMW's have gotten much
more expensive over the years (and GMC to a lesser extent), while
Hyundai's have increased very moderately. Honda (and Kia) seem to have
spiked in 2018 from 2017.

    \hypertarget{question-4-4-points}{%
\section{Question 4 (4 points)}\label{question-4-4-points}}

I recently learned the stereotype that people from Colorado like to
drive Subarus. Does this appear to be true? Calculate appropriate
conditional distributions to assess this claim.

    \begin{Verbatim}[commandchars=\\\{\}]
{\color{incolor}In [{\color{incolor}6}]:} \PY{n}{colorado\PYZus{}cars} \PY{o}{=} \PY{n}{cars\PYZus{}df}\PY{p}{[}\PY{n}{cars\PYZus{}df}\PY{p}{[}\PY{l+s+s2}{\PYZdq{}}\PY{l+s+s2}{State}\PY{l+s+s2}{\PYZdq{}}\PY{p}{]} \PY{o}{==} \PY{l+s+s2}{\PYZdq{}}\PY{l+s+s2}{CO}\PY{l+s+s2}{\PYZdq{}}\PY{p}{]}
        \PY{p}{(}\PY{n}{colorado\PYZus{}cars}\PY{p}{[}\PY{l+s+s2}{\PYZdq{}}\PY{l+s+s2}{Make}\PY{l+s+s2}{\PYZdq{}}\PY{p}{]}\PY{o}{.}\PY{n}{value\PYZus{}counts}\PY{p}{(}\PY{p}{)} \PY{o}{/} 
         \PY{n}{colorado\PYZus{}cars}\PY{p}{[}\PY{l+s+s2}{\PYZdq{}}\PY{l+s+s2}{Make}\PY{l+s+s2}{\PYZdq{}}\PY{p}{]}\PY{o}{.}\PY{n}{count}\PY{p}{(}\PY{p}{)}\PY{p}{)}
\end{Verbatim}


\begin{Verbatim}[commandchars=\\\{\}]
{\color{outcolor}Out[{\color{outcolor}6}]:} FORD             0.142647
        CHEVROLET        0.095588
        TOYOTA           0.077941
        JEEP             0.075000
        NISSAN           0.058088
        SUBARU           0.050000
        HONDA            0.048529
        DODGE            0.038971
        GMC              0.037500
        HYUNDAI          0.034559
        VOLKSWAGEN       0.032353
        BMW              0.032353
        KIA              0.030147
        RAM              0.028676
        AUDI             0.028676
        ACURA            0.022794
        LEXUS            0.020588
        CHRYSLER         0.017647
        CADILLAC         0.017647
        INFINITI         0.017647
        MERCEDES-BENZ    0.017647
        MAZDA            0.011765
        BUICK            0.011029
        MITSUBISHI       0.007353
        MINI             0.005882
        VOLVO            0.005882
        LINCOLN          0.005882
        LAND             0.004412
        PONTIAC          0.002941
        FIAT             0.002941
        MERCURY          0.002206
        PORSCHE          0.002206
        SMART            0.002206
        HUMMER           0.002206
        SATURN           0.002206
        MASERATI         0.001471
        SCION            0.001471
        SUZUKI           0.001471
        JAGUAR           0.000735
        GENESIS          0.000735
        Name: Make, dtype: float64
\end{Verbatim}
            
    By looking at the distribution of car Makes given that the vehicles were
registered in Colorado, we can see that the claim does not hold true. In
fact, the most popular car make in Colorado happens to be Ford. Subaru
is only the 6th popular, accounting for approximately 5\% of the
Colorado vehicles in the data set.

    \hypertarget{question-5-4-points}{%
\section{Question 5 (4 points)}\label{question-5-4-points}}

Calculate the joint distribution between the year and the 10th digit of
the VIN number. What do you notice? Can you explain why this is?

    \begin{Verbatim}[commandchars=\\\{\}]
{\color{incolor}In [{\color{incolor}7}]:} \PY{n}{vin\PYZus{}10\PYZus{}digits} \PY{o}{=} \PY{n}{cars\PYZus{}df}\PY{p}{[}\PY{l+s+s2}{\PYZdq{}}\PY{l+s+s2}{Vin}\PY{l+s+s2}{\PYZdq{}}\PY{p}{]}\PY{o}{.}\PY{n}{str}\PY{o}{.}\PY{n}{get}\PY{p}{(}\PY{l+m+mi}{9}\PY{p}{)}
        \PY{n}{vin\PYZus{}10\PYZus{}digits}
        
        \PY{n}{year\PYZus{}vin\PYZus{}counts} \PY{o}{=} \PY{n}{pd}\PY{o}{.}\PY{n}{crosstab}\PY{p}{(}\PY{n}{cars\PYZus{}df}\PY{o}{.}\PY{n}{Year}\PY{p}{,} \PY{n}{vin\PYZus{}10\PYZus{}digits}\PY{p}{)}
        \PY{p}{(}\PY{n}{year\PYZus{}vin\PYZus{}counts}\PY{o}{.}\PY{n}{divide}\PY{p}{(}
            \PY{n}{year\PYZus{}vin\PYZus{}counts}\PY{o}{.}\PY{n}{sum}\PY{p}{(}\PY{n}{axis}\PY{o}{=}\PY{l+m+mi}{1}\PY{p}{)}\PY{p}{,} \PY{n}{axis}\PY{o}{=}\PY{l+m+mi}{0}\PY{p}{)}\PY{p}{)}
\end{Verbatim}


\begin{Verbatim}[commandchars=\\\{\}]
{\color{outcolor}Out[{\color{outcolor}7}]:} Vin     1    2    3    4    5    6    7    8    9    A {\ldots}          D  \textbackslash{}
        Year                                                   {\ldots}              
        1997  0.0  0.0  0.0  0.0  0.0  0.0  0.0  0.0  0.0  0.0 {\ldots}   0.000000   
        1998  0.0  0.0  0.0  0.0  0.0  0.0  0.0  0.0  0.0  0.0 {\ldots}   0.000000   
        1999  0.0  0.0  0.0  0.0  0.0  0.0  0.0  0.0  0.0  0.0 {\ldots}   0.000000   
        2000  0.0  0.0  0.0  0.0  0.0  0.0  0.0  0.0  0.0  0.0 {\ldots}   0.000000   
        2001  1.0  0.0  0.0  0.0  0.0  0.0  0.0  0.0  0.0  0.0 {\ldots}   0.000000   
        2002  0.0  1.0  0.0  0.0  0.0  0.0  0.0  0.0  0.0  0.0 {\ldots}   0.000000   
        2003  0.0  0.0  1.0  0.0  0.0  0.0  0.0  0.0  0.0  0.0 {\ldots}   0.000000   
        2004  0.0  0.0  0.0  1.0  0.0  0.0  0.0  0.0  0.0  0.0 {\ldots}   0.000000   
        2005  0.0  0.0  0.0  0.0  1.0  0.0  0.0  0.0  0.0  0.0 {\ldots}   0.000000   
        2006  0.0  0.0  0.0  0.0  0.0  1.0  0.0  0.0  0.0  0.0 {\ldots}   0.000000   
        2007  0.0  0.0  0.0  0.0  0.0  0.0  1.0  0.0  0.0  0.0 {\ldots}   0.000000   
        2008  0.0  0.0  0.0  0.0  0.0  0.0  0.0  1.0  0.0  0.0 {\ldots}   0.000000   
        2009  0.0  0.0  0.0  0.0  0.0  0.0  0.0  0.0  1.0  0.0 {\ldots}   0.000000   
        2010  0.0  0.0  0.0  0.0  0.0  0.0  0.0  0.0  0.0  1.0 {\ldots}   0.000000   
        2011  0.0  0.0  0.0  0.0  0.0  0.0  0.0  0.0  0.0  0.0 {\ldots}   0.000000   
        2012  0.0  0.0  0.0  0.0  0.0  0.0  0.0  0.0  0.0  0.0 {\ldots}   0.000000   
        2013  0.0  0.0  0.0  0.0  0.0  0.0  0.0  0.0  0.0  0.0 {\ldots}   1.000000   
        2014  0.0  0.0  0.0  0.0  0.0  0.0  0.0  0.0  0.0  0.0 {\ldots}   0.000105   
        2015  0.0  0.0  0.0  0.0  0.0  0.0  0.0  0.0  0.0  0.0 {\ldots}   0.000000   
        2016  0.0  0.0  0.0  0.0  0.0  0.0  0.0  0.0  0.0  0.0 {\ldots}   0.000000   
        2017  0.0  0.0  0.0  0.0  0.0  0.0  0.0  0.0  0.0  0.0 {\ldots}   0.000000   
        2018  0.0  0.0  0.0  0.0  0.0  0.0  0.0  0.0  0.0  0.0 {\ldots}   0.000000   
        
        Vin          E    F    G    H    J    V    W    X    Y  
        Year                                                    
        1997  0.000000  0.0  0.0  0.0  0.0  1.0  0.0  0.0  0.0  
        1998  0.000000  0.0  0.0  0.0  0.0  0.0  1.0  0.0  0.0  
        1999  0.000000  0.0  0.0  0.0  0.0  0.0  0.0  1.0  0.0  
        2000  0.000000  0.0  0.0  0.0  0.0  0.0  0.0  0.0  1.0  
        2001  0.000000  0.0  0.0  0.0  0.0  0.0  0.0  0.0  0.0  
        2002  0.000000  0.0  0.0  0.0  0.0  0.0  0.0  0.0  0.0  
        2003  0.000000  0.0  0.0  0.0  0.0  0.0  0.0  0.0  0.0  
        2004  0.000000  0.0  0.0  0.0  0.0  0.0  0.0  0.0  0.0  
        2005  0.000000  0.0  0.0  0.0  0.0  0.0  0.0  0.0  0.0  
        2006  0.000000  0.0  0.0  0.0  0.0  0.0  0.0  0.0  0.0  
        2007  0.000000  0.0  0.0  0.0  0.0  0.0  0.0  0.0  0.0  
        2008  0.000000  0.0  0.0  0.0  0.0  0.0  0.0  0.0  0.0  
        2009  0.000000  0.0  0.0  0.0  0.0  0.0  0.0  0.0  0.0  
        2010  0.000000  0.0  0.0  0.0  0.0  0.0  0.0  0.0  0.0  
        2011  0.000000  0.0  0.0  0.0  0.0  0.0  0.0  0.0  0.0  
        2012  0.000000  0.0  0.0  0.0  0.0  0.0  0.0  0.0  0.0  
        2013  0.000000  0.0  0.0  0.0  0.0  0.0  0.0  0.0  0.0  
        2014  0.999895  0.0  0.0  0.0  0.0  0.0  0.0  0.0  0.0  
        2015  0.000000  1.0  0.0  0.0  0.0  0.0  0.0  0.0  0.0  
        2016  0.000000  0.0  1.0  0.0  0.0  0.0  0.0  0.0  0.0  
        2017  0.000000  0.0  0.0  1.0  0.0  0.0  0.0  0.0  0.0  
        2018  0.000000  0.0  0.0  0.0  1.0  0.0  0.0  0.0  0.0  
        
        [22 rows x 22 columns]
\end{Verbatim}
            
    From looking at the joint distribution of year and 10th digit vin, we
can see that there is a correspondence between digit and year of make.
For example, a vehicle made in year 2000 would have a 10th-digit vin of
Y, while one made it 2018 would have a J. This makes sense because my
car was made in 2008, and the 10th digit of its VIN is 8! There are also
missing letters, this is because the manufacturers did not want letters,
such as i, o, u, or z to be confused to others.

    \hypertarget{question-6-4-points}{%
\section{Question 6 (4 points)}\label{question-6-4-points}}

Which states tend to use their cars the most? Calculate the mileage per
year of each car in the data set. (\emph{Reminder:} This data set was
collected in 2018.) Then, make a visualization that shows the average
mileage per year by state, sorted by state. What do you notice?

    \begin{Verbatim}[commandchars=\\\{\}]
{\color{incolor}In [{\color{incolor}19}]:} \PY{n}{cars\PYZus{}df}\PY{p}{[}\PY{l+s+s2}{\PYZdq{}}\PY{l+s+s2}{MPY}\PY{l+s+s2}{\PYZdq{}}\PY{p}{]} \PY{o}{=} \PY{n}{cars\PYZus{}df}\PY{p}{[}\PY{l+s+s2}{\PYZdq{}}\PY{l+s+s2}{Mileage}\PY{l+s+s2}{\PYZdq{}}\PY{p}{]} \PY{o}{/} \PY{p}{(}\PY{l+m+mi}{2018} \PY{o}{\PYZhy{}} \PY{n}{cars\PYZus{}df}\PY{p}{[}\PY{l+s+s2}{\PYZdq{}}\PY{l+s+s2}{Year}\PY{l+s+s2}{\PYZdq{}}\PY{p}{]}\PY{p}{)}
         \PY{n}{cars\PYZus{}df}\PY{o}{.}\PY{n}{loc}\PY{p}{[}\PY{n}{cars\PYZus{}df}\PY{p}{[}\PY{l+s+s2}{\PYZdq{}}\PY{l+s+s2}{Year}\PY{l+s+s2}{\PYZdq{}}\PY{p}{]} \PY{o}{==} \PY{l+m+mi}{2018}\PY{p}{,} \PY{l+s+s2}{\PYZdq{}}\PY{l+s+s2}{MPY}\PY{l+s+s2}{\PYZdq{}}\PY{p}{]} \PY{o}{=} \PY{p}{(}
             \PY{n}{cars\PYZus{}df}\PY{p}{[}\PY{n}{cars\PYZus{}df}\PY{p}{[}\PY{l+s+s2}{\PYZdq{}}\PY{l+s+s2}{Year}\PY{l+s+s2}{\PYZdq{}}\PY{p}{]} \PY{o}{==} \PY{l+m+mi}{2018}\PY{p}{]}\PY{o}{.}\PY{n}{Mileage}\PY{p}{)}
         \PY{p}{(}\PY{n}{cars\PYZus{}df}\PY{o}{.}\PY{n}{groupby}\PY{p}{(}\PY{l+s+s2}{\PYZdq{}}\PY{l+s+s2}{State}\PY{l+s+s2}{\PYZdq{}}\PY{p}{)}\PY{o}{.}\PY{n}{MPY}\PY{o}{.}\PY{n}{mean}\PY{p}{(}\PY{p}{)}
             \PY{o}{.}\PY{n}{plot}\PY{o}{.}\PY{n}{barh}\PY{p}{(}\PY{n}{figsize}\PY{o}{=}\PY{p}{(}\PY{l+m+mi}{20}\PY{p}{,}\PY{l+m+mi}{10}\PY{p}{)}\PY{p}{,} \PY{n}{title}\PY{o}{=}\PY{l+s+s2}{\PYZdq{}}\PY{l+s+s2}{Avg MPY by State}\PY{l+s+s2}{\PYZdq{}}\PY{p}{)}\PY{p}{)}
\end{Verbatim}


\begin{Verbatim}[commandchars=\\\{\}]
{\color{outcolor}Out[{\color{outcolor}19}]:} <matplotlib.axes.\_subplots.AxesSubplot at 0x7f366c362dd8>
\end{Verbatim}
            
    \begin{center}
    \adjustimage{max size={0.9\linewidth}{0.9\paperheight}}{output_19_1.png}
    \end{center}
    { \hspace*{\fill} \\}
    
    From the distribution, we can see that the highest average MPY is in
Montana, and next is Mississippi. The lowest is Hawaii, and California
seems to be pretty average. The distribution does not appear to have any
sort of skew (when sorted by State).

    \hypertarget{question-7-4-points}{%
\section{Question 7 (4 points)}\label{question-7-4-points}}

Suppose you are moving from San Luis Obispo to Houston, TX. You put your
2005 Porsche on sale (observation 8111 in the \texttt{DataFrame}) and
would like to find a similar used car in Houston. Which car on sale in
Houston is most similar to your current car? Is this car a Porsche?

    \begin{Verbatim}[commandchars=\\\{\}]
{\color{incolor}In [{\color{incolor}9}]:} \PY{n}{porsche} \PY{o}{=} \PY{n}{cars\PYZus{}df}\PY{o}{.}\PY{n}{loc}\PY{p}{[}\PY{l+m+mi}{8111}\PY{p}{,} \PY{p}{:}\PY{p}{]}\PY{p}{[}\PY{p}{[}\PY{l+s+s2}{\PYZdq{}}\PY{l+s+s2}{Price}\PY{l+s+s2}{\PYZdq{}}\PY{p}{,} \PY{l+s+s2}{\PYZdq{}}\PY{l+s+s2}{Year}\PY{l+s+s2}{\PYZdq{}}\PY{p}{,} \PY{l+s+s2}{\PYZdq{}}\PY{l+s+s2}{Mileage}\PY{l+s+s2}{\PYZdq{}}\PY{p}{]}\PY{p}{]}
        \PY{n}{tx\PYZus{}cars} \PY{o}{=} \PY{n}{cars\PYZus{}df}\PY{p}{[}\PY{n}{cars\PYZus{}df}\PY{o}{.}\PY{n}{State} \PY{o}{==} \PY{l+s+s2}{\PYZdq{}}\PY{l+s+s2}{TX}\PY{l+s+s2}{\PYZdq{}}\PY{p}{]}
        \PY{n}{houston\PYZus{}cars} \PY{o}{=} \PY{n}{tx\PYZus{}cars}\PY{p}{[}\PY{n}{tx\PYZus{}cars}\PY{o}{.}\PY{n}{City} \PY{o}{==} \PY{l+s+s2}{\PYZdq{}}\PY{l+s+s2}{HOUSTON}\PY{l+s+s2}{\PYZdq{}}\PY{p}{]}\PY{p}{[}\PY{p}{[}\PY{l+s+s2}{\PYZdq{}}\PY{l+s+s2}{Price}\PY{l+s+s2}{\PYZdq{}}\PY{p}{,} \PY{l+s+s2}{\PYZdq{}}\PY{l+s+s2}{Year}\PY{l+s+s2}{\PYZdq{}}\PY{p}{,} \PY{l+s+s2}{\PYZdq{}}\PY{l+s+s2}{Mileage}\PY{l+s+s2}{\PYZdq{}}\PY{p}{]}\PY{p}{]}
        \PY{n}{cars\PYZus{}df}\PY{o}{.}\PY{n}{iloc}\PY{p}{[}\PY{n}{np}\PY{o}{.}\PY{n}{sqrt}\PY{p}{(}\PY{p}{(}\PY{p}{(}\PY{n}{houston\PYZus{}cars} \PY{o}{\PYZhy{}} \PY{n}{porsche}\PY{p}{)} \PY{o}{*}\PY{o}{*} \PY{l+m+mi}{2}\PY{p}{)}\PY{o}{.}\PY{n}{sum}\PY{p}{(}\PY{n}{axis}\PY{o}{=}\PY{l+m+mi}{1}\PY{p}{)}\PY{p}{)}\PY{o}{.}\PY{n}{idxmin}\PY{p}{(}\PY{p}{)}\PY{p}{]}
\end{Verbatim}


\begin{Verbatim}[commandchars=\\\{\}]
{\color{outcolor}Out[{\color{outcolor}9}]:} Price                  33900
        Year                    2012
        Mileage                52650
        City                 HOUSTON
        State                     TX
        Vin        1GNSKBE0XCR161261
        Make               CHEVROLET
        Model               TAHOE4WD
        MPY                     8775
        Name: 41054, dtype: object
\end{Verbatim}
            
    I decided that the most important variables when comparing cars were
price, year and mileage. These trumped other variables such as make and
model, because I decided that those weren't really a consideration,
since I was moving to a much lower cost of living place (the median
house in SLO is nearly 5x that of Houston). I couldn't afford a porsche
any longer!

The Chevrolet Tahoe 4wd was most similar to my current car, based on
these metrics.

    \hypertarget{question-8-4-points}{%
\section{Question 8 (4 points)}\label{question-8-4-points}}

Make a graphic that shows how the last digits of prices and the last
digits of mileages are distributed. Do they appear to be uniformly
distributed over the digits 0-9? How do the two distributions compare to
each other?

    \begin{Verbatim}[commandchars=\\\{\}]
{\color{incolor}In [{\color{incolor}10}]:} \PY{n}{last\PYZus{}digits\PYZus{}price} \PY{o}{=} \PY{n}{cars\PYZus{}df}\PY{o}{.}\PY{n}{Price}\PY{o}{.}\PY{n}{astype}\PY{p}{(}\PY{n+nb}{str}\PY{p}{)}\PY{o}{.}\PY{n}{str}\PY{p}{[}\PY{o}{\PYZhy{}}\PY{l+m+mi}{1}\PY{p}{]}\PY{o}{.}\PY{n}{value\PYZus{}counts}\PY{p}{(}\PY{p}{)}
         \PY{n}{last\PYZus{}digits\PYZus{}price}\PY{o}{.}\PY{n}{sort\PYZus{}index}\PY{p}{(}\PY{n}{inplace}\PY{o}{=}\PY{k+kc}{True}\PY{p}{)}
         \PY{n}{last\PYZus{}digits\PYZus{}mileage} \PY{o}{=} \PY{n}{cars\PYZus{}df}\PY{o}{.}\PY{n}{Mileage}\PY{o}{.}\PY{n}{astype}\PY{p}{(}\PY{n+nb}{str}\PY{p}{)}\PY{o}{.}\PY{n}{str}\PY{p}{[}\PY{o}{\PYZhy{}}\PY{l+m+mi}{1}\PY{p}{]}\PY{o}{.}\PY{n}{value\PYZus{}counts}\PY{p}{(}\PY{p}{)}
         \PY{n}{last\PYZus{}digits\PYZus{}mileage}\PY{o}{.}\PY{n}{sort\PYZus{}index}\PY{p}{(}\PY{n}{inplace}\PY{o}{=}\PY{k+kc}{True}\PY{p}{)}
\end{Verbatim}


    \begin{Verbatim}[commandchars=\\\{\}]
{\color{incolor}In [{\color{incolor}20}]:} \PY{n}{last\PYZus{}digits\PYZus{}price}\PY{o}{.}\PY{n}{plot}\PY{o}{.}\PY{n}{bar}\PY{p}{(}\PY{n}{title}\PY{o}{=}\PY{l+s+s2}{\PYZdq{}}\PY{l+s+s2}{Last Digits \PYZhy{} Price}\PY{l+s+s2}{\PYZdq{}}\PY{p}{)}
\end{Verbatim}


\begin{Verbatim}[commandchars=\\\{\}]
{\color{outcolor}Out[{\color{outcolor}20}]:} <matplotlib.axes.\_subplots.AxesSubplot at 0x7f3673519b38>
\end{Verbatim}
            
    \begin{center}
    \adjustimage{max size={0.9\linewidth}{0.9\paperheight}}{output_26_1.png}
    \end{center}
    { \hspace*{\fill} \\}
    
    \begin{Verbatim}[commandchars=\\\{\}]
{\color{incolor}In [{\color{incolor}21}]:} \PY{n}{last\PYZus{}digits\PYZus{}mileage}\PY{o}{.}\PY{n}{plot}\PY{o}{.}\PY{n}{bar}\PY{p}{(}\PY{n}{title}\PY{o}{=}\PY{l+s+s2}{\PYZdq{}}\PY{l+s+s2}{Last Digits \PYZhy{} Mileage}\PY{l+s+s2}{\PYZdq{}}\PY{p}{)}
\end{Verbatim}


\begin{Verbatim}[commandchars=\\\{\}]
{\color{outcolor}Out[{\color{outcolor}21}]:} <matplotlib.axes.\_subplots.AxesSubplot at 0x7f366c052240>
\end{Verbatim}
            
    \begin{center}
    \adjustimage{max size={0.9\linewidth}{0.9\paperheight}}{output_27_1.png}
    \end{center}
    { \hspace*{\fill} \\}
    
    The disribution of last digits in the prices of cars does not appear to
be uniformly distributed. There is a mucher greater proportion of
observations in 0 and 5. However, the distribution of last digits in
mileages of cars does in fact appear to be (roughly) uniformly
distributed. There does not appear to be a number that dominates the
others as 0 and 5 do in the last digits of price distribution.

    \hypertarget{submission-instructions}{%
\section{Submission Instructions}\label{submission-instructions}}

Once you are finished, follow these steps:

\begin{enumerate}
\def\labelenumi{\arabic{enumi}.}
\tightlist
\item
  Restart the kernel and re-run this notebook from beginning to end by
  going to
  \texttt{Kernel\ \textgreater{}\ Restart\ Kernel\ and\ Run\ All\ Cells}.
\item
  If this process stops halfway through, that means there was an error.
  Correct the error and repeat Step 1 until the notebook runs from
  beginning to end.
\item
  Double check that there is a number next to each code cell and that
  these numbers are in order.
\end{enumerate}

Then, submit your exam as follows:

\begin{enumerate}
\def\labelenumi{\arabic{enumi}.}
\tightlist
\item
  Go to
  \texttt{File\ \textgreater{}\ Export\ Notebook\ As\ \textgreater{}\ PDF}.
\item
  Double check that the entire notebook, from beginning to end, is in
  this PDF file. (If the notebook is cut off, try first exporting the
  notebook to HTML and printing to PDF.)
\item
  Upload the PDF
  \href{https://polylearn.calpoly.edu/AY_2018-2019/mod/assign/view.php?id=319390}{to
  PolyLearn}.
\end{enumerate}


    % Add a bibliography block to the postdoc
    
    
    
    \end{document}
