
% Default to the notebook output style

    


% Inherit from the specified cell style.




    
\documentclass[11pt]{article}

    
    
    \usepackage[T1]{fontenc}
    % Nicer default font (+ math font) than Computer Modern for most use cases
    \usepackage{mathpazo}

    % Basic figure setup, for now with no caption control since it's done
    % automatically by Pandoc (which extracts ![](path) syntax from Markdown).
    \usepackage{graphicx}
    % We will generate all images so they have a width \maxwidth. This means
    % that they will get their normal width if they fit onto the page, but
    % are scaled down if they would overflow the margins.
    \makeatletter
    \def\maxwidth{\ifdim\Gin@nat@width>\linewidth\linewidth
    \else\Gin@nat@width\fi}
    \makeatother
    \let\Oldincludegraphics\includegraphics
    % Set max figure width to be 80% of text width, for now hardcoded.
    \renewcommand{\includegraphics}[1]{\Oldincludegraphics[width=.8\maxwidth]{#1}}
    % Ensure that by default, figures have no caption (until we provide a
    % proper Figure object with a Caption API and a way to capture that
    % in the conversion process - todo).
    \usepackage{caption}
    \DeclareCaptionLabelFormat{nolabel}{}
    \captionsetup{labelformat=nolabel}

    \usepackage{adjustbox} % Used to constrain images to a maximum size 
    \usepackage{xcolor} % Allow colors to be defined
    \usepackage{enumerate} % Needed for markdown enumerations to work
    \usepackage{geometry} % Used to adjust the document margins
    \usepackage{amsmath} % Equations
    \usepackage{amssymb} % Equations
    \usepackage{textcomp} % defines textquotesingle
    % Hack from http://tex.stackexchange.com/a/47451/13684:
    \AtBeginDocument{%
        \def\PYZsq{\textquotesingle}% Upright quotes in Pygmentized code
    }
    \usepackage{upquote} % Upright quotes for verbatim code
    \usepackage{eurosym} % defines \euro
    \usepackage[mathletters]{ucs} % Extended unicode (utf-8) support
    \usepackage[utf8x]{inputenc} % Allow utf-8 characters in the tex document
    \usepackage{fancyvrb} % verbatim replacement that allows latex
    \usepackage{grffile} % extends the file name processing of package graphics 
                         % to support a larger range 
    % The hyperref package gives us a pdf with properly built
    % internal navigation ('pdf bookmarks' for the table of contents,
    % internal cross-reference links, web links for URLs, etc.)
    \usepackage{hyperref}
    \usepackage{longtable} % longtable support required by pandoc >1.10
    \usepackage{booktabs}  % table support for pandoc > 1.12.2
    \usepackage[inline]{enumitem} % IRkernel/repr support (it uses the enumerate* environment)
    \usepackage[normalem]{ulem} % ulem is needed to support strikethroughs (\sout)
                                % normalem makes italics be italics, not underlines
    

    
    
    % Colors for the hyperref package
    \definecolor{urlcolor}{rgb}{0,.145,.698}
    \definecolor{linkcolor}{rgb}{.71,0.21,0.01}
    \definecolor{citecolor}{rgb}{.12,.54,.11}

    % ANSI colors
    \definecolor{ansi-black}{HTML}{3E424D}
    \definecolor{ansi-black-intense}{HTML}{282C36}
    \definecolor{ansi-red}{HTML}{E75C58}
    \definecolor{ansi-red-intense}{HTML}{B22B31}
    \definecolor{ansi-green}{HTML}{00A250}
    \definecolor{ansi-green-intense}{HTML}{007427}
    \definecolor{ansi-yellow}{HTML}{DDB62B}
    \definecolor{ansi-yellow-intense}{HTML}{B27D12}
    \definecolor{ansi-blue}{HTML}{208FFB}
    \definecolor{ansi-blue-intense}{HTML}{0065CA}
    \definecolor{ansi-magenta}{HTML}{D160C4}
    \definecolor{ansi-magenta-intense}{HTML}{A03196}
    \definecolor{ansi-cyan}{HTML}{60C6C8}
    \definecolor{ansi-cyan-intense}{HTML}{258F8F}
    \definecolor{ansi-white}{HTML}{C5C1B4}
    \definecolor{ansi-white-intense}{HTML}{A1A6B2}

    % commands and environments needed by pandoc snippets
    % extracted from the output of `pandoc -s`
    \providecommand{\tightlist}{%
      \setlength{\itemsep}{0pt}\setlength{\parskip}{0pt}}
    \DefineVerbatimEnvironment{Highlighting}{Verbatim}{commandchars=\\\{\}}
    % Add ',fontsize=\small' for more characters per line
    \newenvironment{Shaded}{}{}
    \newcommand{\KeywordTok}[1]{\textcolor[rgb]{0.00,0.44,0.13}{\textbf{{#1}}}}
    \newcommand{\DataTypeTok}[1]{\textcolor[rgb]{0.56,0.13,0.00}{{#1}}}
    \newcommand{\DecValTok}[1]{\textcolor[rgb]{0.25,0.63,0.44}{{#1}}}
    \newcommand{\BaseNTok}[1]{\textcolor[rgb]{0.25,0.63,0.44}{{#1}}}
    \newcommand{\FloatTok}[1]{\textcolor[rgb]{0.25,0.63,0.44}{{#1}}}
    \newcommand{\CharTok}[1]{\textcolor[rgb]{0.25,0.44,0.63}{{#1}}}
    \newcommand{\StringTok}[1]{\textcolor[rgb]{0.25,0.44,0.63}{{#1}}}
    \newcommand{\CommentTok}[1]{\textcolor[rgb]{0.38,0.63,0.69}{\textit{{#1}}}}
    \newcommand{\OtherTok}[1]{\textcolor[rgb]{0.00,0.44,0.13}{{#1}}}
    \newcommand{\AlertTok}[1]{\textcolor[rgb]{1.00,0.00,0.00}{\textbf{{#1}}}}
    \newcommand{\FunctionTok}[1]{\textcolor[rgb]{0.02,0.16,0.49}{{#1}}}
    \newcommand{\RegionMarkerTok}[1]{{#1}}
    \newcommand{\ErrorTok}[1]{\textcolor[rgb]{1.00,0.00,0.00}{\textbf{{#1}}}}
    \newcommand{\NormalTok}[1]{{#1}}
    
    % Additional commands for more recent versions of Pandoc
    \newcommand{\ConstantTok}[1]{\textcolor[rgb]{0.53,0.00,0.00}{{#1}}}
    \newcommand{\SpecialCharTok}[1]{\textcolor[rgb]{0.25,0.44,0.63}{{#1}}}
    \newcommand{\VerbatimStringTok}[1]{\textcolor[rgb]{0.25,0.44,0.63}{{#1}}}
    \newcommand{\SpecialStringTok}[1]{\textcolor[rgb]{0.73,0.40,0.53}{{#1}}}
    \newcommand{\ImportTok}[1]{{#1}}
    \newcommand{\DocumentationTok}[1]{\textcolor[rgb]{0.73,0.13,0.13}{\textit{{#1}}}}
    \newcommand{\AnnotationTok}[1]{\textcolor[rgb]{0.38,0.63,0.69}{\textbf{\textit{{#1}}}}}
    \newcommand{\CommentVarTok}[1]{\textcolor[rgb]{0.38,0.63,0.69}{\textbf{\textit{{#1}}}}}
    \newcommand{\VariableTok}[1]{\textcolor[rgb]{0.10,0.09,0.49}{{#1}}}
    \newcommand{\ControlFlowTok}[1]{\textcolor[rgb]{0.00,0.44,0.13}{\textbf{{#1}}}}
    \newcommand{\OperatorTok}[1]{\textcolor[rgb]{0.40,0.40,0.40}{{#1}}}
    \newcommand{\BuiltInTok}[1]{{#1}}
    \newcommand{\ExtensionTok}[1]{{#1}}
    \newcommand{\PreprocessorTok}[1]{\textcolor[rgb]{0.74,0.48,0.00}{{#1}}}
    \newcommand{\AttributeTok}[1]{\textcolor[rgb]{0.49,0.56,0.16}{{#1}}}
    \newcommand{\InformationTok}[1]{\textcolor[rgb]{0.38,0.63,0.69}{\textbf{\textit{{#1}}}}}
    \newcommand{\WarningTok}[1]{\textcolor[rgb]{0.38,0.63,0.69}{\textbf{\textit{{#1}}}}}
    
    
    % Define a nice break command that doesn't care if a line doesn't already
    % exist.
    \def\br{\hspace*{\fill} \\* }
    % Math Jax compatability definitions
    \def\gt{>}
    \def\lt{<}
    % Document parameters
    \title{2A. Shark Tank}
    
    
    

    % Pygments definitions
    
\makeatletter
\def\PY@reset{\let\PY@it=\relax \let\PY@bf=\relax%
    \let\PY@ul=\relax \let\PY@tc=\relax%
    \let\PY@bc=\relax \let\PY@ff=\relax}
\def\PY@tok#1{\csname PY@tok@#1\endcsname}
\def\PY@toks#1+{\ifx\relax#1\empty\else%
    \PY@tok{#1}\expandafter\PY@toks\fi}
\def\PY@do#1{\PY@bc{\PY@tc{\PY@ul{%
    \PY@it{\PY@bf{\PY@ff{#1}}}}}}}
\def\PY#1#2{\PY@reset\PY@toks#1+\relax+\PY@do{#2}}

\expandafter\def\csname PY@tok@w\endcsname{\def\PY@tc##1{\textcolor[rgb]{0.73,0.73,0.73}{##1}}}
\expandafter\def\csname PY@tok@c\endcsname{\let\PY@it=\textit\def\PY@tc##1{\textcolor[rgb]{0.25,0.50,0.50}{##1}}}
\expandafter\def\csname PY@tok@cp\endcsname{\def\PY@tc##1{\textcolor[rgb]{0.74,0.48,0.00}{##1}}}
\expandafter\def\csname PY@tok@k\endcsname{\let\PY@bf=\textbf\def\PY@tc##1{\textcolor[rgb]{0.00,0.50,0.00}{##1}}}
\expandafter\def\csname PY@tok@kp\endcsname{\def\PY@tc##1{\textcolor[rgb]{0.00,0.50,0.00}{##1}}}
\expandafter\def\csname PY@tok@kt\endcsname{\def\PY@tc##1{\textcolor[rgb]{0.69,0.00,0.25}{##1}}}
\expandafter\def\csname PY@tok@o\endcsname{\def\PY@tc##1{\textcolor[rgb]{0.40,0.40,0.40}{##1}}}
\expandafter\def\csname PY@tok@ow\endcsname{\let\PY@bf=\textbf\def\PY@tc##1{\textcolor[rgb]{0.67,0.13,1.00}{##1}}}
\expandafter\def\csname PY@tok@nb\endcsname{\def\PY@tc##1{\textcolor[rgb]{0.00,0.50,0.00}{##1}}}
\expandafter\def\csname PY@tok@nf\endcsname{\def\PY@tc##1{\textcolor[rgb]{0.00,0.00,1.00}{##1}}}
\expandafter\def\csname PY@tok@nc\endcsname{\let\PY@bf=\textbf\def\PY@tc##1{\textcolor[rgb]{0.00,0.00,1.00}{##1}}}
\expandafter\def\csname PY@tok@nn\endcsname{\let\PY@bf=\textbf\def\PY@tc##1{\textcolor[rgb]{0.00,0.00,1.00}{##1}}}
\expandafter\def\csname PY@tok@ne\endcsname{\let\PY@bf=\textbf\def\PY@tc##1{\textcolor[rgb]{0.82,0.25,0.23}{##1}}}
\expandafter\def\csname PY@tok@nv\endcsname{\def\PY@tc##1{\textcolor[rgb]{0.10,0.09,0.49}{##1}}}
\expandafter\def\csname PY@tok@no\endcsname{\def\PY@tc##1{\textcolor[rgb]{0.53,0.00,0.00}{##1}}}
\expandafter\def\csname PY@tok@nl\endcsname{\def\PY@tc##1{\textcolor[rgb]{0.63,0.63,0.00}{##1}}}
\expandafter\def\csname PY@tok@ni\endcsname{\let\PY@bf=\textbf\def\PY@tc##1{\textcolor[rgb]{0.60,0.60,0.60}{##1}}}
\expandafter\def\csname PY@tok@na\endcsname{\def\PY@tc##1{\textcolor[rgb]{0.49,0.56,0.16}{##1}}}
\expandafter\def\csname PY@tok@nt\endcsname{\let\PY@bf=\textbf\def\PY@tc##1{\textcolor[rgb]{0.00,0.50,0.00}{##1}}}
\expandafter\def\csname PY@tok@nd\endcsname{\def\PY@tc##1{\textcolor[rgb]{0.67,0.13,1.00}{##1}}}
\expandafter\def\csname PY@tok@s\endcsname{\def\PY@tc##1{\textcolor[rgb]{0.73,0.13,0.13}{##1}}}
\expandafter\def\csname PY@tok@sd\endcsname{\let\PY@it=\textit\def\PY@tc##1{\textcolor[rgb]{0.73,0.13,0.13}{##1}}}
\expandafter\def\csname PY@tok@si\endcsname{\let\PY@bf=\textbf\def\PY@tc##1{\textcolor[rgb]{0.73,0.40,0.53}{##1}}}
\expandafter\def\csname PY@tok@se\endcsname{\let\PY@bf=\textbf\def\PY@tc##1{\textcolor[rgb]{0.73,0.40,0.13}{##1}}}
\expandafter\def\csname PY@tok@sr\endcsname{\def\PY@tc##1{\textcolor[rgb]{0.73,0.40,0.53}{##1}}}
\expandafter\def\csname PY@tok@ss\endcsname{\def\PY@tc##1{\textcolor[rgb]{0.10,0.09,0.49}{##1}}}
\expandafter\def\csname PY@tok@sx\endcsname{\def\PY@tc##1{\textcolor[rgb]{0.00,0.50,0.00}{##1}}}
\expandafter\def\csname PY@tok@m\endcsname{\def\PY@tc##1{\textcolor[rgb]{0.40,0.40,0.40}{##1}}}
\expandafter\def\csname PY@tok@gh\endcsname{\let\PY@bf=\textbf\def\PY@tc##1{\textcolor[rgb]{0.00,0.00,0.50}{##1}}}
\expandafter\def\csname PY@tok@gu\endcsname{\let\PY@bf=\textbf\def\PY@tc##1{\textcolor[rgb]{0.50,0.00,0.50}{##1}}}
\expandafter\def\csname PY@tok@gd\endcsname{\def\PY@tc##1{\textcolor[rgb]{0.63,0.00,0.00}{##1}}}
\expandafter\def\csname PY@tok@gi\endcsname{\def\PY@tc##1{\textcolor[rgb]{0.00,0.63,0.00}{##1}}}
\expandafter\def\csname PY@tok@gr\endcsname{\def\PY@tc##1{\textcolor[rgb]{1.00,0.00,0.00}{##1}}}
\expandafter\def\csname PY@tok@ge\endcsname{\let\PY@it=\textit}
\expandafter\def\csname PY@tok@gs\endcsname{\let\PY@bf=\textbf}
\expandafter\def\csname PY@tok@gp\endcsname{\let\PY@bf=\textbf\def\PY@tc##1{\textcolor[rgb]{0.00,0.00,0.50}{##1}}}
\expandafter\def\csname PY@tok@go\endcsname{\def\PY@tc##1{\textcolor[rgb]{0.53,0.53,0.53}{##1}}}
\expandafter\def\csname PY@tok@gt\endcsname{\def\PY@tc##1{\textcolor[rgb]{0.00,0.27,0.87}{##1}}}
\expandafter\def\csname PY@tok@err\endcsname{\def\PY@bc##1{\setlength{\fboxsep}{0pt}\fcolorbox[rgb]{1.00,0.00,0.00}{1,1,1}{\strut ##1}}}
\expandafter\def\csname PY@tok@kc\endcsname{\let\PY@bf=\textbf\def\PY@tc##1{\textcolor[rgb]{0.00,0.50,0.00}{##1}}}
\expandafter\def\csname PY@tok@kd\endcsname{\let\PY@bf=\textbf\def\PY@tc##1{\textcolor[rgb]{0.00,0.50,0.00}{##1}}}
\expandafter\def\csname PY@tok@kn\endcsname{\let\PY@bf=\textbf\def\PY@tc##1{\textcolor[rgb]{0.00,0.50,0.00}{##1}}}
\expandafter\def\csname PY@tok@kr\endcsname{\let\PY@bf=\textbf\def\PY@tc##1{\textcolor[rgb]{0.00,0.50,0.00}{##1}}}
\expandafter\def\csname PY@tok@bp\endcsname{\def\PY@tc##1{\textcolor[rgb]{0.00,0.50,0.00}{##1}}}
\expandafter\def\csname PY@tok@fm\endcsname{\def\PY@tc##1{\textcolor[rgb]{0.00,0.00,1.00}{##1}}}
\expandafter\def\csname PY@tok@vc\endcsname{\def\PY@tc##1{\textcolor[rgb]{0.10,0.09,0.49}{##1}}}
\expandafter\def\csname PY@tok@vg\endcsname{\def\PY@tc##1{\textcolor[rgb]{0.10,0.09,0.49}{##1}}}
\expandafter\def\csname PY@tok@vi\endcsname{\def\PY@tc##1{\textcolor[rgb]{0.10,0.09,0.49}{##1}}}
\expandafter\def\csname PY@tok@vm\endcsname{\def\PY@tc##1{\textcolor[rgb]{0.10,0.09,0.49}{##1}}}
\expandafter\def\csname PY@tok@sa\endcsname{\def\PY@tc##1{\textcolor[rgb]{0.73,0.13,0.13}{##1}}}
\expandafter\def\csname PY@tok@sb\endcsname{\def\PY@tc##1{\textcolor[rgb]{0.73,0.13,0.13}{##1}}}
\expandafter\def\csname PY@tok@sc\endcsname{\def\PY@tc##1{\textcolor[rgb]{0.73,0.13,0.13}{##1}}}
\expandafter\def\csname PY@tok@dl\endcsname{\def\PY@tc##1{\textcolor[rgb]{0.73,0.13,0.13}{##1}}}
\expandafter\def\csname PY@tok@s2\endcsname{\def\PY@tc##1{\textcolor[rgb]{0.73,0.13,0.13}{##1}}}
\expandafter\def\csname PY@tok@sh\endcsname{\def\PY@tc##1{\textcolor[rgb]{0.73,0.13,0.13}{##1}}}
\expandafter\def\csname PY@tok@s1\endcsname{\def\PY@tc##1{\textcolor[rgb]{0.73,0.13,0.13}{##1}}}
\expandafter\def\csname PY@tok@mb\endcsname{\def\PY@tc##1{\textcolor[rgb]{0.40,0.40,0.40}{##1}}}
\expandafter\def\csname PY@tok@mf\endcsname{\def\PY@tc##1{\textcolor[rgb]{0.40,0.40,0.40}{##1}}}
\expandafter\def\csname PY@tok@mh\endcsname{\def\PY@tc##1{\textcolor[rgb]{0.40,0.40,0.40}{##1}}}
\expandafter\def\csname PY@tok@mi\endcsname{\def\PY@tc##1{\textcolor[rgb]{0.40,0.40,0.40}{##1}}}
\expandafter\def\csname PY@tok@il\endcsname{\def\PY@tc##1{\textcolor[rgb]{0.40,0.40,0.40}{##1}}}
\expandafter\def\csname PY@tok@mo\endcsname{\def\PY@tc##1{\textcolor[rgb]{0.40,0.40,0.40}{##1}}}
\expandafter\def\csname PY@tok@ch\endcsname{\let\PY@it=\textit\def\PY@tc##1{\textcolor[rgb]{0.25,0.50,0.50}{##1}}}
\expandafter\def\csname PY@tok@cm\endcsname{\let\PY@it=\textit\def\PY@tc##1{\textcolor[rgb]{0.25,0.50,0.50}{##1}}}
\expandafter\def\csname PY@tok@cpf\endcsname{\let\PY@it=\textit\def\PY@tc##1{\textcolor[rgb]{0.25,0.50,0.50}{##1}}}
\expandafter\def\csname PY@tok@c1\endcsname{\let\PY@it=\textit\def\PY@tc##1{\textcolor[rgb]{0.25,0.50,0.50}{##1}}}
\expandafter\def\csname PY@tok@cs\endcsname{\let\PY@it=\textit\def\PY@tc##1{\textcolor[rgb]{0.25,0.50,0.50}{##1}}}

\def\PYZbs{\char`\\}
\def\PYZus{\char`\_}
\def\PYZob{\char`\{}
\def\PYZcb{\char`\}}
\def\PYZca{\char`\^}
\def\PYZam{\char`\&}
\def\PYZlt{\char`\<}
\def\PYZgt{\char`\>}
\def\PYZsh{\char`\#}
\def\PYZpc{\char`\%}
\def\PYZdl{\char`\$}
\def\PYZhy{\char`\-}
\def\PYZsq{\char`\'}
\def\PYZdq{\char`\"}
\def\PYZti{\char`\~}
% for compatibility with earlier versions
\def\PYZat{@}
\def\PYZlb{[}
\def\PYZrb{]}
\makeatother


    % Exact colors from NB
    \definecolor{incolor}{rgb}{0.0, 0.0, 0.5}
    \definecolor{outcolor}{rgb}{0.545, 0.0, 0.0}



    
    % Prevent overflowing lines due to hard-to-break entities
    \sloppy 
    % Setup hyperref package
    \hypersetup{
      breaklinks=true,  % so long urls are correctly broken across lines
      colorlinks=true,
      urlcolor=urlcolor,
      linkcolor=linkcolor,
      citecolor=citecolor,
      }
    % Slightly bigger margins than the latex defaults
    
    \geometry{verbose,tmargin=1in,bmargin=1in,lmargin=1in,rmargin=1in}
    
    

    \begin{document}
    
    
    \maketitle
    
    

    
    \hypertarget{shark-tank}{%
\section{Shark Tank}\label{shark-tank}}

\emph{Shark Tank} is a reality TV show. Contestants present their idea
for a company to a panel of investors (a.k.a. ``sharks''), who then
decide whether or not to invest in that company. The investors give a
certain amount of money in exchange for a percentage stake in the
company (``equity''). If you are not familiar with the show, you may
want to watch part of an episode
\href{http://abc.go.com/shows/shark-tank}{here} to get a sense of how it
works.

The data that you will examine in this lab contains data about all
contestants from the first 6 seasons of the show, including: - the name
and industry of the proposed company - whether or not it was funded
(i.e., the ``Deal'' column) - which sharks chose to invest in the
venture (N.B. There are 7 regular sharks, not including ``Guest''. Each
shark has a column in the data set, labeled by their last name.) - if
funded, the amount of money the sharks put in and the percentage equity
they got in return

To earn full credit on this lab, you should: - use built-in
\texttt{pandas} methods (like \texttt{.sum()} and \texttt{.max()})
instead of writing a for loop over a \texttt{DataFrame} or
\texttt{Series} - use the split-apply-combine pattern wherever possible

Of course, if you can't think of a vectorized solution, a \texttt{for}
loop is still better than no solution at all!

    \begin{Verbatim}[commandchars=\\\{\}]
{\color{incolor}In [{\color{incolor}1}]:} \PY{k+kn}{import} \PY{n+nn}{pandas} \PY{k}{as} \PY{n+nn}{pd}
        \PY{o}{\PYZpc{}}\PY{k}{matplotlib} inline
\end{Verbatim}


    \hypertarget{question-0.-getting-and-cleaning-the-data}{%
\subsection{Question 0. Getting and Cleaning the
Data}\label{question-0.-getting-and-cleaning-the-data}}

    The data is stored in the CSV file
\texttt{https://raw.githubusercontent.com/dlsun/data-science-book/master/data/sharktank.csv}.
Read in the data into a Pandas \texttt{DataFrame}.

    \begin{Verbatim}[commandchars=\\\{\}]
{\color{incolor}In [{\color{incolor}2}]:} \PY{n}{df} \PY{o}{=} \PY{n}{pd}\PY{o}{.}\PY{n}{read\PYZus{}csv}\PY{p}{(}\PY{l+s+s2}{\PYZdq{}}\PY{l+s+s2}{https://raw.githubusercontent.com/dlsun/data\PYZhy{}science\PYZhy{}book/master/data/sharktank.csv}\PY{l+s+s2}{\PYZdq{}}\PY{p}{)}
\end{Verbatim}


    There is one column for each of the sharks. A 1 indicates that they
chose to invest in that company, while a missing value indicates that
they did not choose to invest in that company. Notice that these missing
values show up as NaNs when we read in the data. Fill in these missing
values with zeros. Other columns may also contain NaNs; be careful not
to fill those columns with zeros, or you may end up with strange results
down the line.

    \begin{Verbatim}[commandchars=\\\{\}]
{\color{incolor}In [{\color{incolor}3}]:} \PY{n}{df}\PY{o}{.}\PY{n}{iloc}\PY{p}{[}\PY{p}{:}\PY{p}{,} \PY{l+m+mi}{8}\PY{p}{:}\PY{l+m+mi}{16}\PY{p}{]} \PY{o}{=} \PY{n}{df}\PY{o}{.}\PY{n}{iloc}\PY{p}{[}\PY{p}{:}\PY{p}{,}\PY{l+m+mi}{8}\PY{p}{:}\PY{l+m+mi}{16}\PY{p}{]}\PY{o}{.}\PY{n}{fillna}\PY{p}{(}\PY{l+m+mi}{0}\PY{p}{)}
\end{Verbatim}


    Notice that Amount and Equity are currently being treated as categorical
variables (\texttt{dtype:\ object}). Can you figure out why this is?
Clean up these columns and cast them to numeric types (i.e., a
\texttt{dtype} of \texttt{int} or \texttt{float}) because we'll need to
perform mathematical operations on these columns.

    \begin{Verbatim}[commandchars=\\\{\}]
{\color{incolor}In [{\color{incolor}4}]:} \PY{n}{df}\PY{o}{.}\PY{n}{Equity} \PY{o}{=} \PY{p}{(}\PY{n}{df}\PY{o}{.}\PY{n}{Equity}\PY{o}{.}\PY{n}{str}\PY{o}{.}\PY{n}{replace}\PY{p}{(}\PY{l+s+s2}{\PYZdq{}}\PY{l+s+s2}{\PYZpc{}}\PY{l+s+s2}{\PYZdq{}}\PY{p}{,} \PY{l+s+s2}{\PYZdq{}}\PY{l+s+s2}{\PYZdq{}}\PY{p}{)}\PY{o}{.}\PY{n}{astype}\PY{p}{(}\PY{n+nb}{float}\PY{p}{)}\PY{o}{/}\PY{l+m+mi}{100}\PY{p}{)}\PY{o}{.}\PY{n}{fillna}\PY{p}{(}\PY{l+m+mi}{0}\PY{p}{)}
        \PY{n}{df}\PY{o}{.}\PY{n}{Amount} \PY{o}{=} \PY{p}{(}\PY{n}{df}\PY{o}{.}\PY{n}{Amount}\PY{o}{.}\PY{n}{str}\PY{o}{.}\PY{n}{replace}\PY{p}{(}\PY{l+s+sa}{r}\PY{l+s+s2}{\PYZdq{}}\PY{l+s+s2}{[\PYZca{}0\PYZhy{}9]}\PY{l+s+s2}{\PYZdq{}}\PY{p}{,} \PY{l+s+s2}{\PYZdq{}}\PY{l+s+s2}{\PYZdq{}}\PY{p}{)}\PY{o}{.}\PY{n}{astype}\PY{p}{(}\PY{n+nb}{float}\PY{p}{)}\PY{p}{)}\PY{o}{.}\PY{n}{fillna}\PY{p}{(}\PY{l+m+mi}{0}\PY{p}{)}
        \PY{n}{df}\PY{o}{.}\PY{n}{head}\PY{p}{(}\PY{p}{)}
\end{Verbatim}


\begin{Verbatim}[commandchars=\\\{\}]
{\color{outcolor}Out[{\color{outcolor}4}]:}    Season  No. in series                      Company Deal           Industry  \textbackslash{}
        0     1.0            1.0             Ava the Elephant  Yes         Healthcare   
        1     1.0            1.0        Mr. Tod's Pie Factory  Yes  Food and Beverage   
        2     1.0            1.0                      Wispots   No  Business Services   
        3     1.0            1.0  College Foxes Packing Boxes   No   Lifestyle / Home   
        4     1.0            1.0                    Ionic Ear   No  Uncertain / Other   
        
          Entrepreneur Gender    Amount  Equity  Corcoran  Cuban  Greiner  Herjavec  \textbackslash{}
        0              Female   50000.0    0.55       1.0    0.0      0.0       0.0   
        1                Male  460000.0    0.50       1.0    0.0      0.0       0.0   
        2                Male       0.0    0.00       0.0    0.0      0.0       0.0   
        3                Male       0.0    0.00       0.0    0.0      0.0       0.0   
        4                Male       0.0    0.00       0.0    0.0      0.0       0.0   
        
           John  O'Leary  Harrington  Guest Details / Notes  
        0   0.0      0.0         0.0    0.0             NaN  
        1   1.0      0.0         0.0    0.0             NaN  
        2   0.0      0.0         0.0    0.0             NaN  
        3   0.0      0.0         0.0    0.0             NaN  
        4   0.0      0.0         0.0    0.0             NaN  
\end{Verbatim}
            
    \hypertarget{question-1.-which-company-was-worth-the-most}{%
\subsection{Question 1. Which Company was Worth the
Most?}\label{question-1.-which-company-was-worth-the-most}}

    The valuation of a company is how much it is worth. If someone invests
\$10,000 for a 40\% equity stake in the company, then this means the
company must be valued at \$25,000, since 40\% of \$25,000 is \$10,000.

Calculate the valuation of each company that was funded. Which company
was most valuable? Is it the same as the company that received the
largest total investment from the sharks?

    \begin{Verbatim}[commandchars=\\\{\}]
{\color{incolor}In [{\color{incolor}5}]:} \PY{k+kn}{import} \PY{n+nn}{numpy} \PY{k}{as} \PY{n+nn}{np}
        \PY{n}{valuations} \PY{o}{=} \PY{n}{df}\PY{o}{.}\PY{n}{Amount} \PY{o}{/} \PY{n}{df}\PY{o}{.}\PY{n}{Equity}
        \PY{n}{valuations} \PY{o}{=} \PY{n}{valuations}\PY{o}{.}\PY{n}{replace}\PY{p}{(}\PY{p}{[}\PY{n}{np}\PY{o}{.}\PY{n}{inf}\PY{p}{]}\PY{p}{,} \PY{l+m+mi}{0}\PY{p}{)}
        \PY{n+nb}{print}\PY{p}{(}\PY{n}{df}\PY{o}{.}\PY{n}{iloc}\PY{p}{[}\PY{n}{valuations}\PY{o}{.}\PY{n}{idxmax}\PY{p}{(}\PY{p}{)}\PY{p}{,} \PY{p}{:}\PY{p}{]}\PY{p}{)}
        \PY{n+nb}{print}\PY{p}{(}\PY{n}{df}\PY{o}{.}\PY{n}{iloc}\PY{p}{[}\PY{n}{df}\PY{o}{.}\PY{n}{Amount}\PY{o}{.}\PY{n}{idxmax}\PY{p}{(}\PY{p}{)}\PY{p}{,} \PY{p}{:}\PY{p}{]}\PY{p}{)}
\end{Verbatim}


    \begin{Verbatim}[commandchars=\\\{\}]
Season                                                                 6
No. in series                                                         11
Company                                                             Zipz
Deal                                                                 Yes
Industry                                               Food and Beverage
Entrepreneur Gender                                                 Male
Amount                                                           2.5e+06
Equity                                                               0.1
Corcoran                                                               0
Cuban                                                                  0
Greiner                                                                0
Herjavec                                                               0
John                                                                   0
O'Leary                                                                1
Harrington                                                             0
Guest                                                                  0
Details / Notes        with an option for another \$2.5 Million for an{\ldots}
Name: 421, dtype: object
Season                                                                 6
No. in series                                                         27
Company                                                           AirCar
Deal                                                                 Yes
Industry                                                 Green/CleanTech
Entrepreneur Gender                                                 Male
Amount                                                             5e+06
Equity                                                               0.5
Corcoran                                                               0
Cuban                                                                  0
Greiner                                                                0
Herjavec                                                               1
John                                                                   0
O'Leary                                                                0
Harrington                                                             0
Guest                                                                  0
Details / Notes        Contingent on getting deal to bring to contine{\ldots}
Name: 483, dtype: object

    \end{Verbatim}

    I decided to disregard data points that had ``Infinite'' value, or those
which had investment without explicit equity stated.

The most valuable company in the Shark's eyes was Zipz. AirCar received
the largest total investment from the Sharks.

    \hypertarget{question-2.-which-shark-invested-the-most}{%
\subsection{Question 2. Which Shark Invested the
Most?}\label{question-2.-which-shark-invested-the-most}}

    Calculate the total amount of money that each shark invested over the 6
seasons. Which shark invested the most total money over the 6 seasons?

\emph{Hint:} If \(n\) sharks funded a given venture, then the amount
that each shark invested is the total amount divided by \(n\).

    \begin{Verbatim}[commandchars=\\\{\}]
{\color{incolor}In [{\color{incolor}6}]:} \PY{p}{(}\PY{n}{df}\PY{o}{.}\PY{n}{iloc}\PY{p}{[}\PY{p}{:}\PY{p}{,} \PY{l+m+mi}{8}\PY{p}{:}\PY{l+m+mi}{16}\PY{p}{]}\PY{o}{.}\PY{n}{multiply}\PY{p}{(}\PY{n}{df}\PY{o}{.}\PY{n}{Amount} \PY{o}{/} \PY{n}{df}\PY{o}{.}\PY{n}{iloc}\PY{p}{[}\PY{p}{:}\PY{p}{,} \PY{l+m+mi}{8}\PY{p}{:}\PY{l+m+mi}{16}\PY{p}{]}
                                   \PY{o}{.}\PY{n}{astype}\PY{p}{(}\PY{n+nb}{int}\PY{p}{)}\PY{o}{.}\PY{n}{sum}\PY{p}{(}\PY{n}{axis}\PY{o}{=}\PY{l+m+mi}{1}\PY{p}{)}\PY{p}{,} \PY{n}{axis}\PY{o}{=}\PY{l+m+mi}{0}\PY{p}{)}\PY{o}{.}\PY{n}{sum}\PY{p}{(}\PY{p}{)}\PY{p}{)}
\end{Verbatim}


\begin{Verbatim}[commandchars=\\\{\}]
{\color{outcolor}Out[{\color{outcolor}6}]:} Corcoran       4912500.0
        Cuban         17817500.0
        Greiner        8170000.0
        Herjavec      16297500.0
        John           8154000.0
        O'Leary        7952500.0
        Harrington      800000.0
        Guest           400000.0
        dtype: float64
\end{Verbatim}
            
    I decided to include guest, because if a guest invested in a situation,
the shark's investments were lower than if the guest had not invested.

However, the shark that invested the most money was Cuban, with a net
investment of \$17,817,500!

    \hypertarget{question-3.-do-the-sharks-prefer-certain-industries}{%
\subsection{Question 3. Do the Sharks Prefer Certain
Industries?}\label{question-3.-do-the-sharks-prefer-certain-industries}}

Calculate the funding rate (the proportion of companies that were
funded) for each industry. Make a visualization showing this
information.

    \begin{Verbatim}[commandchars=\\\{\}]
{\color{incolor}In [{\color{incolor}7}]:} \PY{p}{(}\PY{n}{df}\PY{o}{.}\PY{n}{groupby}\PY{p}{(}\PY{l+s+s2}{\PYZdq{}}\PY{l+s+s2}{Industry}\PY{l+s+s2}{\PYZdq{}}\PY{p}{)}\PY{o}{.}\PY{n}{Deal}\PY{o}{.}\PY{n}{value\PYZus{}counts}\PY{p}{(}\PY{n}{normalize}\PY{o}{=}\PY{k+kc}{True}\PY{p}{)}\PY{o}{.}\PY{n}{to\PYZus{}frame}\PY{p}{(}\PY{p}{)}
                                    \PY{o}{.}\PY{n}{unstack}\PY{p}{(}\PY{p}{)}\PY{o}{.}\PY{n}{Deal}\PY{p}{[}\PY{l+s+s1}{\PYZsq{}}\PY{l+s+s1}{Yes}\PY{l+s+s1}{\PYZsq{}}\PY{p}{]}\PY{o}{.}\PY{n}{plot}\PY{o}{.}\PY{n}{barh}\PY{p}{(}\PY{p}{)}\PY{p}{)}
\end{Verbatim}


\begin{Verbatim}[commandchars=\\\{\}]
{\color{outcolor}Out[{\color{outcolor}7}]:} <matplotlib.axes.\_subplots.AxesSubplot at 0x7f5c6cfb27f0>
\end{Verbatim}
            
    \begin{center}
    \adjustimage{max size={0.9\linewidth}{0.9\paperheight}}{output_18_1.png}
    \end{center}
    { \hspace*{\fill} \\}
    
    This was a toughy. I decided to get the value counts (normalized) for
the Deal variable by Industry. I converted the resulting series into a
dataframe, changed deal from index to column (with unstack), and then
got all the `Yes' proportions from that.

    \hypertarget{submission-instructions}{%
\subsection{Submission Instructions}\label{submission-instructions}}

Once you are finished, follow these steps:

\begin{enumerate}
\def\labelenumi{\arabic{enumi}.}
\tightlist
\item
  Restart the kernel and re-run this notebook from beginning to end by
  going to
  \texttt{Kernel\ \textgreater{}\ Restart\ Kernel\ and\ Run\ All\ Cells}.
\item
  If this process stops halfway through, that means there was an error.
  Correct the error and repeat Step 1 until the notebook runs from
  beginning to end.
\item
  Double check that there is a number next to each code cell and that
  these numbers are in order.
\end{enumerate}

Then, submit your lab as follows:

\begin{enumerate}
\def\labelenumi{\arabic{enumi}.}
\tightlist
\item
  Go to
  \texttt{File\ \textgreater{}\ Export\ Notebook\ As\ \textgreater{}\ PDF}.
\item
  Double check that the entire notebook, from beginning to end, is in
  this PDF file. (If the notebook is cut off, try first exporting the
  notebook to HTML and printing to PDF.)
\item
  Upload the PDF
  \href{https://polylearn.calpoly.edu/AY_2018-2019/mod/assign/view.php?id=306677}{to
  PolyLearn}.
\end{enumerate}


    % Add a bibliography block to the postdoc
    
    
    
    \end{document}
