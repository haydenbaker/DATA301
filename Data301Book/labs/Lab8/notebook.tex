
% Default to the notebook output style

    


% Inherit from the specified cell style.




    
\documentclass[11pt]{article}

    
    
    \usepackage[T1]{fontenc}
    % Nicer default font (+ math font) than Computer Modern for most use cases
    \usepackage{mathpazo}

    % Basic figure setup, for now with no caption control since it's done
    % automatically by Pandoc (which extracts ![](path) syntax from Markdown).
    \usepackage{graphicx}
    % We will generate all images so they have a width \maxwidth. This means
    % that they will get their normal width if they fit onto the page, but
    % are scaled down if they would overflow the margins.
    \makeatletter
    \def\maxwidth{\ifdim\Gin@nat@width>\linewidth\linewidth
    \else\Gin@nat@width\fi}
    \makeatother
    \let\Oldincludegraphics\includegraphics
    % Set max figure width to be 80% of text width, for now hardcoded.
    \renewcommand{\includegraphics}[1]{\Oldincludegraphics[width=.8\maxwidth]{#1}}
    % Ensure that by default, figures have no caption (until we provide a
    % proper Figure object with a Caption API and a way to capture that
    % in the conversion process - todo).
    \usepackage{caption}
    \DeclareCaptionLabelFormat{nolabel}{}
    \captionsetup{labelformat=nolabel}

    \usepackage{adjustbox} % Used to constrain images to a maximum size 
    \usepackage{xcolor} % Allow colors to be defined
    \usepackage{enumerate} % Needed for markdown enumerations to work
    \usepackage{geometry} % Used to adjust the document margins
    \usepackage{amsmath} % Equations
    \usepackage{amssymb} % Equations
    \usepackage{textcomp} % defines textquotesingle
    % Hack from http://tex.stackexchange.com/a/47451/13684:
    \AtBeginDocument{%
        \def\PYZsq{\textquotesingle}% Upright quotes in Pygmentized code
    }
    \usepackage{upquote} % Upright quotes for verbatim code
    \usepackage{eurosym} % defines \euro
    \usepackage[mathletters]{ucs} % Extended unicode (utf-8) support
    \usepackage[utf8x]{inputenc} % Allow utf-8 characters in the tex document
    \usepackage{fancyvrb} % verbatim replacement that allows latex
    \usepackage{grffile} % extends the file name processing of package graphics 
                         % to support a larger range 
    % The hyperref package gives us a pdf with properly built
    % internal navigation ('pdf bookmarks' for the table of contents,
    % internal cross-reference links, web links for URLs, etc.)
    \usepackage{hyperref}
    \usepackage{longtable} % longtable support required by pandoc >1.10
    \usepackage{booktabs}  % table support for pandoc > 1.12.2
    \usepackage[inline]{enumitem} % IRkernel/repr support (it uses the enumerate* environment)
    \usepackage[normalem]{ulem} % ulem is needed to support strikethroughs (\sout)
                                % normalem makes italics be italics, not underlines
    

    
    
    % Colors for the hyperref package
    \definecolor{urlcolor}{rgb}{0,.145,.698}
    \definecolor{linkcolor}{rgb}{.71,0.21,0.01}
    \definecolor{citecolor}{rgb}{.12,.54,.11}

    % ANSI colors
    \definecolor{ansi-black}{HTML}{3E424D}
    \definecolor{ansi-black-intense}{HTML}{282C36}
    \definecolor{ansi-red}{HTML}{E75C58}
    \definecolor{ansi-red-intense}{HTML}{B22B31}
    \definecolor{ansi-green}{HTML}{00A250}
    \definecolor{ansi-green-intense}{HTML}{007427}
    \definecolor{ansi-yellow}{HTML}{DDB62B}
    \definecolor{ansi-yellow-intense}{HTML}{B27D12}
    \definecolor{ansi-blue}{HTML}{208FFB}
    \definecolor{ansi-blue-intense}{HTML}{0065CA}
    \definecolor{ansi-magenta}{HTML}{D160C4}
    \definecolor{ansi-magenta-intense}{HTML}{A03196}
    \definecolor{ansi-cyan}{HTML}{60C6C8}
    \definecolor{ansi-cyan-intense}{HTML}{258F8F}
    \definecolor{ansi-white}{HTML}{C5C1B4}
    \definecolor{ansi-white-intense}{HTML}{A1A6B2}

    % commands and environments needed by pandoc snippets
    % extracted from the output of `pandoc -s`
    \providecommand{\tightlist}{%
      \setlength{\itemsep}{0pt}\setlength{\parskip}{0pt}}
    \DefineVerbatimEnvironment{Highlighting}{Verbatim}{commandchars=\\\{\}}
    % Add ',fontsize=\small' for more characters per line
    \newenvironment{Shaded}{}{}
    \newcommand{\KeywordTok}[1]{\textcolor[rgb]{0.00,0.44,0.13}{\textbf{{#1}}}}
    \newcommand{\DataTypeTok}[1]{\textcolor[rgb]{0.56,0.13,0.00}{{#1}}}
    \newcommand{\DecValTok}[1]{\textcolor[rgb]{0.25,0.63,0.44}{{#1}}}
    \newcommand{\BaseNTok}[1]{\textcolor[rgb]{0.25,0.63,0.44}{{#1}}}
    \newcommand{\FloatTok}[1]{\textcolor[rgb]{0.25,0.63,0.44}{{#1}}}
    \newcommand{\CharTok}[1]{\textcolor[rgb]{0.25,0.44,0.63}{{#1}}}
    \newcommand{\StringTok}[1]{\textcolor[rgb]{0.25,0.44,0.63}{{#1}}}
    \newcommand{\CommentTok}[1]{\textcolor[rgb]{0.38,0.63,0.69}{\textit{{#1}}}}
    \newcommand{\OtherTok}[1]{\textcolor[rgb]{0.00,0.44,0.13}{{#1}}}
    \newcommand{\AlertTok}[1]{\textcolor[rgb]{1.00,0.00,0.00}{\textbf{{#1}}}}
    \newcommand{\FunctionTok}[1]{\textcolor[rgb]{0.02,0.16,0.49}{{#1}}}
    \newcommand{\RegionMarkerTok}[1]{{#1}}
    \newcommand{\ErrorTok}[1]{\textcolor[rgb]{1.00,0.00,0.00}{\textbf{{#1}}}}
    \newcommand{\NormalTok}[1]{{#1}}
    
    % Additional commands for more recent versions of Pandoc
    \newcommand{\ConstantTok}[1]{\textcolor[rgb]{0.53,0.00,0.00}{{#1}}}
    \newcommand{\SpecialCharTok}[1]{\textcolor[rgb]{0.25,0.44,0.63}{{#1}}}
    \newcommand{\VerbatimStringTok}[1]{\textcolor[rgb]{0.25,0.44,0.63}{{#1}}}
    \newcommand{\SpecialStringTok}[1]{\textcolor[rgb]{0.73,0.40,0.53}{{#1}}}
    \newcommand{\ImportTok}[1]{{#1}}
    \newcommand{\DocumentationTok}[1]{\textcolor[rgb]{0.73,0.13,0.13}{\textit{{#1}}}}
    \newcommand{\AnnotationTok}[1]{\textcolor[rgb]{0.38,0.63,0.69}{\textbf{\textit{{#1}}}}}
    \newcommand{\CommentVarTok}[1]{\textcolor[rgb]{0.38,0.63,0.69}{\textbf{\textit{{#1}}}}}
    \newcommand{\VariableTok}[1]{\textcolor[rgb]{0.10,0.09,0.49}{{#1}}}
    \newcommand{\ControlFlowTok}[1]{\textcolor[rgb]{0.00,0.44,0.13}{\textbf{{#1}}}}
    \newcommand{\OperatorTok}[1]{\textcolor[rgb]{0.40,0.40,0.40}{{#1}}}
    \newcommand{\BuiltInTok}[1]{{#1}}
    \newcommand{\ExtensionTok}[1]{{#1}}
    \newcommand{\PreprocessorTok}[1]{\textcolor[rgb]{0.74,0.48,0.00}{{#1}}}
    \newcommand{\AttributeTok}[1]{\textcolor[rgb]{0.49,0.56,0.16}{{#1}}}
    \newcommand{\InformationTok}[1]{\textcolor[rgb]{0.38,0.63,0.69}{\textbf{\textit{{#1}}}}}
    \newcommand{\WarningTok}[1]{\textcolor[rgb]{0.38,0.63,0.69}{\textbf{\textit{{#1}}}}}
    
    
    % Define a nice break command that doesn't care if a line doesn't already
    % exist.
    \def\br{\hspace*{\fill} \\* }
    % Math Jax compatability definitions
    \def\gt{>}
    \def\lt{<}
    % Document parameters
    \title{8A. Song Lyrics Generator}
    
    
    

    % Pygments definitions
    
\makeatletter
\def\PY@reset{\let\PY@it=\relax \let\PY@bf=\relax%
    \let\PY@ul=\relax \let\PY@tc=\relax%
    \let\PY@bc=\relax \let\PY@ff=\relax}
\def\PY@tok#1{\csname PY@tok@#1\endcsname}
\def\PY@toks#1+{\ifx\relax#1\empty\else%
    \PY@tok{#1}\expandafter\PY@toks\fi}
\def\PY@do#1{\PY@bc{\PY@tc{\PY@ul{%
    \PY@it{\PY@bf{\PY@ff{#1}}}}}}}
\def\PY#1#2{\PY@reset\PY@toks#1+\relax+\PY@do{#2}}

\expandafter\def\csname PY@tok@w\endcsname{\def\PY@tc##1{\textcolor[rgb]{0.73,0.73,0.73}{##1}}}
\expandafter\def\csname PY@tok@c\endcsname{\let\PY@it=\textit\def\PY@tc##1{\textcolor[rgb]{0.25,0.50,0.50}{##1}}}
\expandafter\def\csname PY@tok@cp\endcsname{\def\PY@tc##1{\textcolor[rgb]{0.74,0.48,0.00}{##1}}}
\expandafter\def\csname PY@tok@k\endcsname{\let\PY@bf=\textbf\def\PY@tc##1{\textcolor[rgb]{0.00,0.50,0.00}{##1}}}
\expandafter\def\csname PY@tok@kp\endcsname{\def\PY@tc##1{\textcolor[rgb]{0.00,0.50,0.00}{##1}}}
\expandafter\def\csname PY@tok@kt\endcsname{\def\PY@tc##1{\textcolor[rgb]{0.69,0.00,0.25}{##1}}}
\expandafter\def\csname PY@tok@o\endcsname{\def\PY@tc##1{\textcolor[rgb]{0.40,0.40,0.40}{##1}}}
\expandafter\def\csname PY@tok@ow\endcsname{\let\PY@bf=\textbf\def\PY@tc##1{\textcolor[rgb]{0.67,0.13,1.00}{##1}}}
\expandafter\def\csname PY@tok@nb\endcsname{\def\PY@tc##1{\textcolor[rgb]{0.00,0.50,0.00}{##1}}}
\expandafter\def\csname PY@tok@nf\endcsname{\def\PY@tc##1{\textcolor[rgb]{0.00,0.00,1.00}{##1}}}
\expandafter\def\csname PY@tok@nc\endcsname{\let\PY@bf=\textbf\def\PY@tc##1{\textcolor[rgb]{0.00,0.00,1.00}{##1}}}
\expandafter\def\csname PY@tok@nn\endcsname{\let\PY@bf=\textbf\def\PY@tc##1{\textcolor[rgb]{0.00,0.00,1.00}{##1}}}
\expandafter\def\csname PY@tok@ne\endcsname{\let\PY@bf=\textbf\def\PY@tc##1{\textcolor[rgb]{0.82,0.25,0.23}{##1}}}
\expandafter\def\csname PY@tok@nv\endcsname{\def\PY@tc##1{\textcolor[rgb]{0.10,0.09,0.49}{##1}}}
\expandafter\def\csname PY@tok@no\endcsname{\def\PY@tc##1{\textcolor[rgb]{0.53,0.00,0.00}{##1}}}
\expandafter\def\csname PY@tok@nl\endcsname{\def\PY@tc##1{\textcolor[rgb]{0.63,0.63,0.00}{##1}}}
\expandafter\def\csname PY@tok@ni\endcsname{\let\PY@bf=\textbf\def\PY@tc##1{\textcolor[rgb]{0.60,0.60,0.60}{##1}}}
\expandafter\def\csname PY@tok@na\endcsname{\def\PY@tc##1{\textcolor[rgb]{0.49,0.56,0.16}{##1}}}
\expandafter\def\csname PY@tok@nt\endcsname{\let\PY@bf=\textbf\def\PY@tc##1{\textcolor[rgb]{0.00,0.50,0.00}{##1}}}
\expandafter\def\csname PY@tok@nd\endcsname{\def\PY@tc##1{\textcolor[rgb]{0.67,0.13,1.00}{##1}}}
\expandafter\def\csname PY@tok@s\endcsname{\def\PY@tc##1{\textcolor[rgb]{0.73,0.13,0.13}{##1}}}
\expandafter\def\csname PY@tok@sd\endcsname{\let\PY@it=\textit\def\PY@tc##1{\textcolor[rgb]{0.73,0.13,0.13}{##1}}}
\expandafter\def\csname PY@tok@si\endcsname{\let\PY@bf=\textbf\def\PY@tc##1{\textcolor[rgb]{0.73,0.40,0.53}{##1}}}
\expandafter\def\csname PY@tok@se\endcsname{\let\PY@bf=\textbf\def\PY@tc##1{\textcolor[rgb]{0.73,0.40,0.13}{##1}}}
\expandafter\def\csname PY@tok@sr\endcsname{\def\PY@tc##1{\textcolor[rgb]{0.73,0.40,0.53}{##1}}}
\expandafter\def\csname PY@tok@ss\endcsname{\def\PY@tc##1{\textcolor[rgb]{0.10,0.09,0.49}{##1}}}
\expandafter\def\csname PY@tok@sx\endcsname{\def\PY@tc##1{\textcolor[rgb]{0.00,0.50,0.00}{##1}}}
\expandafter\def\csname PY@tok@m\endcsname{\def\PY@tc##1{\textcolor[rgb]{0.40,0.40,0.40}{##1}}}
\expandafter\def\csname PY@tok@gh\endcsname{\let\PY@bf=\textbf\def\PY@tc##1{\textcolor[rgb]{0.00,0.00,0.50}{##1}}}
\expandafter\def\csname PY@tok@gu\endcsname{\let\PY@bf=\textbf\def\PY@tc##1{\textcolor[rgb]{0.50,0.00,0.50}{##1}}}
\expandafter\def\csname PY@tok@gd\endcsname{\def\PY@tc##1{\textcolor[rgb]{0.63,0.00,0.00}{##1}}}
\expandafter\def\csname PY@tok@gi\endcsname{\def\PY@tc##1{\textcolor[rgb]{0.00,0.63,0.00}{##1}}}
\expandafter\def\csname PY@tok@gr\endcsname{\def\PY@tc##1{\textcolor[rgb]{1.00,0.00,0.00}{##1}}}
\expandafter\def\csname PY@tok@ge\endcsname{\let\PY@it=\textit}
\expandafter\def\csname PY@tok@gs\endcsname{\let\PY@bf=\textbf}
\expandafter\def\csname PY@tok@gp\endcsname{\let\PY@bf=\textbf\def\PY@tc##1{\textcolor[rgb]{0.00,0.00,0.50}{##1}}}
\expandafter\def\csname PY@tok@go\endcsname{\def\PY@tc##1{\textcolor[rgb]{0.53,0.53,0.53}{##1}}}
\expandafter\def\csname PY@tok@gt\endcsname{\def\PY@tc##1{\textcolor[rgb]{0.00,0.27,0.87}{##1}}}
\expandafter\def\csname PY@tok@err\endcsname{\def\PY@bc##1{\setlength{\fboxsep}{0pt}\fcolorbox[rgb]{1.00,0.00,0.00}{1,1,1}{\strut ##1}}}
\expandafter\def\csname PY@tok@kc\endcsname{\let\PY@bf=\textbf\def\PY@tc##1{\textcolor[rgb]{0.00,0.50,0.00}{##1}}}
\expandafter\def\csname PY@tok@kd\endcsname{\let\PY@bf=\textbf\def\PY@tc##1{\textcolor[rgb]{0.00,0.50,0.00}{##1}}}
\expandafter\def\csname PY@tok@kn\endcsname{\let\PY@bf=\textbf\def\PY@tc##1{\textcolor[rgb]{0.00,0.50,0.00}{##1}}}
\expandafter\def\csname PY@tok@kr\endcsname{\let\PY@bf=\textbf\def\PY@tc##1{\textcolor[rgb]{0.00,0.50,0.00}{##1}}}
\expandafter\def\csname PY@tok@bp\endcsname{\def\PY@tc##1{\textcolor[rgb]{0.00,0.50,0.00}{##1}}}
\expandafter\def\csname PY@tok@fm\endcsname{\def\PY@tc##1{\textcolor[rgb]{0.00,0.00,1.00}{##1}}}
\expandafter\def\csname PY@tok@vc\endcsname{\def\PY@tc##1{\textcolor[rgb]{0.10,0.09,0.49}{##1}}}
\expandafter\def\csname PY@tok@vg\endcsname{\def\PY@tc##1{\textcolor[rgb]{0.10,0.09,0.49}{##1}}}
\expandafter\def\csname PY@tok@vi\endcsname{\def\PY@tc##1{\textcolor[rgb]{0.10,0.09,0.49}{##1}}}
\expandafter\def\csname PY@tok@vm\endcsname{\def\PY@tc##1{\textcolor[rgb]{0.10,0.09,0.49}{##1}}}
\expandafter\def\csname PY@tok@sa\endcsname{\def\PY@tc##1{\textcolor[rgb]{0.73,0.13,0.13}{##1}}}
\expandafter\def\csname PY@tok@sb\endcsname{\def\PY@tc##1{\textcolor[rgb]{0.73,0.13,0.13}{##1}}}
\expandafter\def\csname PY@tok@sc\endcsname{\def\PY@tc##1{\textcolor[rgb]{0.73,0.13,0.13}{##1}}}
\expandafter\def\csname PY@tok@dl\endcsname{\def\PY@tc##1{\textcolor[rgb]{0.73,0.13,0.13}{##1}}}
\expandafter\def\csname PY@tok@s2\endcsname{\def\PY@tc##1{\textcolor[rgb]{0.73,0.13,0.13}{##1}}}
\expandafter\def\csname PY@tok@sh\endcsname{\def\PY@tc##1{\textcolor[rgb]{0.73,0.13,0.13}{##1}}}
\expandafter\def\csname PY@tok@s1\endcsname{\def\PY@tc##1{\textcolor[rgb]{0.73,0.13,0.13}{##1}}}
\expandafter\def\csname PY@tok@mb\endcsname{\def\PY@tc##1{\textcolor[rgb]{0.40,0.40,0.40}{##1}}}
\expandafter\def\csname PY@tok@mf\endcsname{\def\PY@tc##1{\textcolor[rgb]{0.40,0.40,0.40}{##1}}}
\expandafter\def\csname PY@tok@mh\endcsname{\def\PY@tc##1{\textcolor[rgb]{0.40,0.40,0.40}{##1}}}
\expandafter\def\csname PY@tok@mi\endcsname{\def\PY@tc##1{\textcolor[rgb]{0.40,0.40,0.40}{##1}}}
\expandafter\def\csname PY@tok@il\endcsname{\def\PY@tc##1{\textcolor[rgb]{0.40,0.40,0.40}{##1}}}
\expandafter\def\csname PY@tok@mo\endcsname{\def\PY@tc##1{\textcolor[rgb]{0.40,0.40,0.40}{##1}}}
\expandafter\def\csname PY@tok@ch\endcsname{\let\PY@it=\textit\def\PY@tc##1{\textcolor[rgb]{0.25,0.50,0.50}{##1}}}
\expandafter\def\csname PY@tok@cm\endcsname{\let\PY@it=\textit\def\PY@tc##1{\textcolor[rgb]{0.25,0.50,0.50}{##1}}}
\expandafter\def\csname PY@tok@cpf\endcsname{\let\PY@it=\textit\def\PY@tc##1{\textcolor[rgb]{0.25,0.50,0.50}{##1}}}
\expandafter\def\csname PY@tok@c1\endcsname{\let\PY@it=\textit\def\PY@tc##1{\textcolor[rgb]{0.25,0.50,0.50}{##1}}}
\expandafter\def\csname PY@tok@cs\endcsname{\let\PY@it=\textit\def\PY@tc##1{\textcolor[rgb]{0.25,0.50,0.50}{##1}}}

\def\PYZbs{\char`\\}
\def\PYZus{\char`\_}
\def\PYZob{\char`\{}
\def\PYZcb{\char`\}}
\def\PYZca{\char`\^}
\def\PYZam{\char`\&}
\def\PYZlt{\char`\<}
\def\PYZgt{\char`\>}
\def\PYZsh{\char`\#}
\def\PYZpc{\char`\%}
\def\PYZdl{\char`\$}
\def\PYZhy{\char`\-}
\def\PYZsq{\char`\'}
\def\PYZdq{\char`\"}
\def\PYZti{\char`\~}
% for compatibility with earlier versions
\def\PYZat{@}
\def\PYZlb{[}
\def\PYZrb{]}
\makeatother


    % Exact colors from NB
    \definecolor{incolor}{rgb}{0.0, 0.0, 0.5}
    \definecolor{outcolor}{rgb}{0.545, 0.0, 0.0}



    
    % Prevent overflowing lines due to hard-to-break entities
    \sloppy 
    % Setup hyperref package
    \hypersetup{
      breaklinks=true,  % so long urls are correctly broken across lines
      colorlinks=true,
      urlcolor=urlcolor,
      linkcolor=linkcolor,
      citecolor=citecolor,
      }
    % Slightly bigger margins than the latex defaults
    
    \geometry{verbose,tmargin=1in,bmargin=1in,lmargin=1in,rmargin=1in}
    
    

    \begin{document}
    
    
    \maketitle
    
    

    
    \hypertarget{a.-song-lyrics-generator}{%
\section{8A. Song Lyrics Generator}\label{a.-song-lyrics-generator}}

In this lab, you will scrape a website to get lyrics of songs by your
favorite artist. Then, you will train a model called a Markov chain on
these lyrics so that you can generate a song in the style of your
favorite artist.

\hypertarget{question-1.-scraping-song-lyrics}{%
\section{Question 1. Scraping Song
Lyrics}\label{question-1.-scraping-song-lyrics}}

Find a web site that has lyrics for several songs by your favorite
artist. Scrape the lyrics into a Python list called \texttt{lyrics},
where each element of the list represents the lyrics of one song.

\textbf{Tips:} - Find a web page that has links to all of the songs,
like \href{http://www.azlyrics.com/n/nirvana.html}{this one}.
{[}\emph{Note:} It appears that \texttt{azlyrics.com} blocks web
scraping, so you'll have to find a different lyrics web site.{]} Then,
you can scrape this page, extract the hyperlinks, and issue new HTTP
requests to each hyperlink to get each song. - Use \texttt{time.sleep()}
to stagger your HTTP requests so that you do not get banned by the
website for making too many requests.

    \begin{Verbatim}[commandchars=\\\{\}]
{\color{incolor}In [{\color{incolor}1}]:} \PY{k+kn}{import} \PY{n+nn}{requests}\PY{o}{,} \PY{n+nn}{re}\PY{o}{,} \PY{n+nn}{time}\PY{o}{,} \PY{n+nn}{pandas} \PY{k}{as} \PY{n+nn}{pd}
        \PY{k+kn}{from} \PY{n+nn}{bs4} \PY{k}{import} \PY{n}{BeautifulSoup}
        
        \PY{k}{def} \PY{n+nf}{depaginate}\PY{p}{(}\PY{n}{url}\PY{p}{)}\PY{p}{:}
            \PY{c+c1}{\PYZsh{}https://genius.com/api/artists/330928/songs?sort=title\PYZam{}page=1}
            \PY{n}{resps} \PY{o}{=} \PY{p}{[}\PY{p}{]}
            \PY{n}{next\PYZus{}page} \PY{o}{=} \PY{l+s+s2}{\PYZdq{}}\PY{l+s+s2}{\PYZdq{}}
            \PY{n}{resp} \PY{o}{=} \PY{n}{requests}\PY{o}{.}\PY{n}{get}\PY{p}{(}\PY{n}{url} \PY{o}{+} \PY{l+s+s2}{\PYZdq{}}\PY{l+s+s2}{\PYZam{}page=}\PY{l+s+s2}{\PYZdq{}} \PY{o}{+} \PY{l+s+s2}{\PYZdq{}}\PY{l+s+s2}{1}\PY{l+s+s2}{\PYZdq{}}\PY{p}{)}\PY{o}{.}\PY{n}{json}\PY{p}{(}\PY{p}{)}
            \PY{k}{while} \PY{n}{next\PYZus{}page} \PY{o+ow}{is} \PY{o+ow}{not} \PY{k+kc}{None}\PY{p}{:}
                \PY{n}{resps}\PY{o}{.}\PY{n}{append}\PY{p}{(}\PY{n}{resp}\PY{p}{)}
                \PY{n}{next\PYZus{}page} \PY{o}{=} \PY{n}{resp}\PY{p}{[}\PY{l+s+s2}{\PYZdq{}}\PY{l+s+s2}{response}\PY{l+s+s2}{\PYZdq{}}\PY{p}{]}\PY{p}{[}\PY{l+s+s2}{\PYZdq{}}\PY{l+s+s2}{next\PYZus{}page}\PY{l+s+s2}{\PYZdq{}}\PY{p}{]}
                \PY{n}{resp} \PY{o}{=} \PY{n}{requests}\PY{o}{.}\PY{n}{get}\PY{p}{(}\PY{n}{url} \PY{o}{+} \PY{l+s+s2}{\PYZdq{}}\PY{l+s+s2}{\PYZam{}page=}\PY{l+s+s2}{\PYZdq{}} \PY{o}{+} \PY{n+nb}{str}\PY{p}{(}\PY{n}{next\PYZus{}page}\PY{p}{)}\PY{p}{)}\PY{o}{.}\PY{n}{json}\PY{p}{(}\PY{p}{)}
                \PY{n}{time}\PY{o}{.}\PY{n}{sleep}\PY{p}{(}\PY{l+m+mf}{0.25}\PY{p}{)}
            \PY{k}{return} \PY{n}{resps}
        
        \PY{k}{def} \PY{n+nf}{parseLyric}\PY{p}{(}\PY{n}{string}\PY{p}{)}\PY{p}{:}
            \PY{n}{lyric} \PY{o}{=} \PY{n}{re}\PY{o}{.}\PY{n}{sub}\PY{p}{(}\PY{l+s+s2}{\PYZdq{}}\PY{l+s+s2}{\PYZbs{}}\PY{l+s+s2}{[([A\PYZhy{}Za\PYZhy{}z0\PYZhy{}9\PYZus{} \PYZhy{}}\PY{l+s+s2}{\PYZbs{}}\PY{l+s+s2}{?]+)}\PY{l+s+s2}{\PYZbs{}}\PY{l+s+s2}{]}\PY{l+s+s2}{\PYZdq{}}\PY{p}{,} \PY{l+s+s2}{\PYZdq{}}\PY{l+s+s2}{\PYZdq{}}\PY{p}{,} \PY{n}{string}\PY{p}{)}
            \PY{n}{lyric} \PY{o}{=} \PY{n}{lyric}\PY{o}{.}\PY{n}{strip}\PY{p}{(}\PY{p}{)}
            \PY{k}{return} \PY{n}{re}\PY{o}{.}\PY{n}{sub}\PY{p}{(}\PY{l+s+s2}{\PYZdq{}}\PY{l+s+s2}{[}\PY{l+s+se}{\PYZbs{}\PYZbs{}}\PY{l+s+s2}{()}\PY{l+s+se}{\PYZbs{}\PYZdq{}}\PY{l+s+s2}{]}\PY{l+s+s2}{\PYZdq{}}\PY{p}{,} \PY{l+s+s2}{\PYZdq{}}\PY{l+s+s2}{\PYZdq{}}\PY{p}{,} \PY{n}{lyric}\PY{o}{.}\PY{n}{replace}\PY{p}{(}\PY{l+s+s2}{\PYZdq{}}\PY{l+s+se}{\PYZbs{}n}\PY{l+s+s2}{\PYZdq{}}\PY{p}{,} \PY{l+s+s2}{\PYZdq{}}\PY{l+s+s2}{ \PYZlt{}N\PYZgt{} }\PY{l+s+s2}{\PYZdq{}}\PY{p}{)}\PY{p}{)}
        
        \PY{k}{def} \PY{n+nf}{getLyrics}\PY{p}{(}\PY{n}{songs}\PY{p}{)}\PY{p}{:}
            \PY{n}{lyrics} \PY{o}{=} \PY{p}{[}\PY{p}{]}
            \PY{k}{for} \PY{n}{song} \PY{o+ow}{in} \PY{n}{songs}\PY{p}{:}
                \PY{n}{html} \PY{o}{=} \PY{n}{requests}\PY{o}{.}\PY{n}{get}\PY{p}{(}\PY{l+s+s2}{\PYZdq{}}\PY{l+s+s2}{https://genius.com}\PY{l+s+s2}{\PYZdq{}} \PY{o}{+} \PY{n}{song}\PY{p}{[}\PY{l+s+s2}{\PYZdq{}}\PY{l+s+s2}{path}\PY{l+s+s2}{\PYZdq{}}\PY{p}{]}\PY{p}{)}
                \PY{n}{soup} \PY{o}{=} \PY{n}{BeautifulSoup}\PY{p}{(}\PY{n}{html}\PY{o}{.}\PY{n}{content}\PY{p}{,} \PY{l+s+s2}{\PYZdq{}}\PY{l+s+s2}{html.parser}\PY{l+s+s2}{\PYZdq{}}\PY{p}{)}
                \PY{n}{lyrics}\PY{o}{.}\PY{n}{append}\PY{p}{(}\PY{n}{parseLyric}\PY{p}{(}\PY{n}{soup}\PY{o}{.}\PY{n}{find}\PY{p}{(}\PY{l+s+s2}{\PYZdq{}}\PY{l+s+s2}{div}\PY{l+s+s2}{\PYZdq{}}\PY{p}{,} \PY{n}{class\PYZus{}}\PY{o}{=}\PY{l+s+s2}{\PYZdq{}}\PY{l+s+s2}{lyrics}\PY{l+s+s2}{\PYZdq{}}\PY{p}{)}\PY{o}{.}\PY{n}{p}\PY{o}{.}\PY{n}{text}\PY{p}{)}\PY{p}{)}
            \PY{k}{return} \PY{n}{lyrics}
\end{Verbatim}


    \begin{Verbatim}[commandchars=\\\{\}]
{\color{incolor}In [{\color{incolor}2}]:} \PY{n}{lyrics} \PY{o}{=} \PY{p}{[}\PY{p}{]}
        \PY{k}{for} \PY{n}{page} \PY{o+ow}{in} \PY{n}{depaginate}\PY{p}{(}
            \PY{l+s+s2}{\PYZdq{}}\PY{l+s+s2}{https://genius.com/api/artists/330928/songs?sort=title}\PY{l+s+s2}{\PYZdq{}}\PY{p}{)}\PY{p}{:}
            \PY{n}{lyrics} \PY{o}{+}\PY{o}{=} \PY{n}{getLyrics}\PY{p}{(}\PY{n}{page}\PY{p}{[}\PY{l+s+s2}{\PYZdq{}}\PY{l+s+s2}{response}\PY{l+s+s2}{\PYZdq{}}\PY{p}{]}\PY{p}{[}\PY{l+s+s2}{\PYZdq{}}\PY{l+s+s2}{songs}\PY{l+s+s2}{\PYZdq{}}\PY{p}{]}\PY{p}{)}
\end{Verbatim}


    \begin{Verbatim}[commandchars=\\\{\}]
{\color{incolor}In [{\color{incolor}3}]:} \PY{c+c1}{\PYZsh{} Print out the lyrics to the first song.}
        \PY{n+nb}{print}\PY{p}{(}\PY{n}{lyrics}\PY{p}{[}\PY{l+m+mi}{0}\PY{p}{]}\PY{p}{)}
\end{Verbatim}


    \begin{Verbatim}[commandchars=\\\{\}]
Just forget it <N> I can make it mind over matter <N> I can make you feel something better <N> I don't wanna be apart <N> And I said it <N> Everything in life has an ending <N> Think of all the days we were spending <N> Hoping for a way to start <N>  <N>  <N> I came to leave it right <N> I hope to stay the night <N> You back away but it's already love <N>  <N>  <N> I know you think that I'm crazy <N> Cause I can't stop calling you baby <N> And I know that you'll never break me <N> Cause it's already love <N> Cause it's already love <N>  <N>  <N> I'm a soldier <N> Fight it, but I know that I need it <N> Try to look away but I see ya <N> Yeah, your body feels like home <N> Is it over? <N> We don't have to wait on forever <N> We can go a long way together <N> I just wanna let you know <N>  <N>  <N> I came to leave it right <N> I hope to stay the night <N> You back away but it's already love <N>  <N>  <N> I know you think that I'm crazy <N> Cause I can't stop calling you baby <N> And I know that you'll never break me <N> Cause it's already love <N> Cause it's already love <N> You know that look that you gave me <N> Yeah it knocked me out, but it saved me <N> And I don't quite know what I'm saying <N> Cause it's already love <N> Cause it's already love <N>  <N>  <N> I feel it coming back <N> Hoping for a little <N> Waiting for somebody to come <N> I'll show you what to knock for <N> Nothing can mistake my power <N> Everything you want is everything I wanted <N> Now I wanna tell you, now I wanna tell you <N>  <N>  <N> I know you think that I'm crazy <N> Cause I can't stop calling you baby <N> And I know that you'll never break me <N> Cause it's already love <N> Cause it's already love <N> You know that look that you gave me <N> Yeah it knocked me out, but it saved me <N> And I don't quite know what I'm saying <N> Cause it's already love <N> Cause it's already love

    \end{Verbatim}

    \begin{Verbatim}[commandchars=\\\{\}]
{\color{incolor}In [{\color{incolor}4}]:} \PY{n}{lyrics}\PY{p}{[}\PY{l+m+mi}{0}\PY{p}{]}\PY{o}{.}\PY{n}{split}\PY{p}{(}\PY{p}{)}\PY{p}{[}\PY{l+m+mi}{0}\PY{p}{:}\PY{l+m+mi}{8}\PY{p}{]}
\end{Verbatim}


\begin{Verbatim}[commandchars=\\\{\}]
{\color{outcolor}Out[{\color{outcolor}4}]:} ['Just', 'forget', 'it', '<N>', 'I', 'can', 'make', 'it']
\end{Verbatim}
            
    \texttt{pickle} is a Python library that serializes Python objects to
disk so that you can load them in later.

    \begin{Verbatim}[commandchars=\\\{\}]
{\color{incolor}In [{\color{incolor}5}]:} \PY{k+kn}{import} \PY{n+nn}{pickle}
        \PY{n}{pickle}\PY{o}{.}\PY{n}{dump}\PY{p}{(}\PY{n}{lyrics}\PY{p}{,} \PY{n+nb}{open}\PY{p}{(}\PY{l+s+s2}{\PYZdq{}}\PY{l+s+s2}{lyrics.pkl}\PY{l+s+s2}{\PYZdq{}}\PY{p}{,} \PY{l+s+s2}{\PYZdq{}}\PY{l+s+s2}{wb}\PY{l+s+s2}{\PYZdq{}}\PY{p}{)}\PY{p}{)}
\end{Verbatim}


    \hypertarget{question-2.-unigram-markov-chain-model}{%
\section{Question 2. Unigram Markov Chain
Model}\label{question-2.-unigram-markov-chain-model}}

You will build a Markov chain for the artist whose lyrics you scraped in
Lab A. Your model will process the lyrics and store the word transitions
for that artist. The transitions will be stored in a dict called
\texttt{chain}, which maps each word to a list of ``next'' words.

For example, if your song was
\href{https://www.youtube.com/watch?v=FgDU17xqNXo}{``The Joker'' by the
Steve Miller Band}, \texttt{chain} might look as follows:

\begin{verbatim}
chain = {
    "some": ["people", "call", "people"],
    "call": ["me", "me", "me"],
    "the": ["space", "gangster", "pompitous", ...],
    "me": ["the", "the", "Maurice"],
    ...
}
\end{verbatim}

Besides words, you should include a few additional states in your Markov
chain. You should have \texttt{"\textless{}START\textgreater{}"} and
\texttt{"\textless{}END\textgreater{}"} states so that we can keep track
of how songs are likely to begin and end. You should also include a
state called \texttt{"\textless{}N\textgreater{}"} to denote line breaks
so that you can keep track of where lines begin and end. It is up to you
whether you want to include normalize case and strip punctuation.

So for example, for
\href{https://www.azlyrics.com/lyrics/stevemillerband/thejoker.html}{``The
Joker''}, you would add the following to your chain:

\begin{verbatim}
chain = {
    "<START>": ["Some", ...],
    "Some": ["people", ...],
    "people": ["call", ...],
    "call": ["me", ...],
    "me": ["the", ...],
    "the": ["space", ...],
    "space": ["cowboy,", ...],
    "cowboy,": ["yeah", ...],
    "yeah": ["<N>", ...],
    "<N>": ["Some", ..., "Come"],
    ...,
    "Come": ["on", ...],
    "on": ["baby", ...],
    "baby": ["and", ...],
    "and": ["I'll", ...],
    "I'll": ["show", ...],
    "show": ["you", ...],
    "you": ["a", ...],
    "a": ["good", ...],
    "good": ["time", ...],
    "time": ["<END>", ...],
}
\end{verbatim}

Your chain will be trained on not just one song, but by all songs by
your artist.

    \begin{Verbatim}[commandchars=\\\{\}]
{\color{incolor}In [{\color{incolor}15}]:} \PY{k}{def} \PY{n+nf}{train\PYZus{}markov\PYZus{}chain}\PY{p}{(}\PY{n}{lyrics}\PY{p}{)}\PY{p}{:}
             \PY{l+s+sd}{\PYZdq{}\PYZdq{}\PYZdq{}}
         \PY{l+s+sd}{    Args:}
         \PY{l+s+sd}{      \PYZhy{} lyrics: a list of strings, where each string represents}
         \PY{l+s+sd}{                the lyrics of one song by an artist.}
         \PY{l+s+sd}{    }
         \PY{l+s+sd}{    Returns:}
         \PY{l+s+sd}{      A dict that maps a single word (\PYZdq{}unigram\PYZdq{}) to a list of}
         \PY{l+s+sd}{      words that follow that word, representing the Markov}
         \PY{l+s+sd}{      chain trained on the lyrics.}
         \PY{l+s+sd}{    \PYZdq{}\PYZdq{}\PYZdq{}}
             \PY{n}{chain} \PY{o}{=} \PY{p}{\PYZob{}}\PY{l+s+s2}{\PYZdq{}}\PY{l+s+s2}{\PYZlt{}START\PYZgt{}}\PY{l+s+s2}{\PYZdq{}}\PY{p}{:} \PY{p}{[}\PY{p}{]}\PY{p}{\PYZcb{}}
             \PY{k}{for} \PY{n}{lyric} \PY{o+ow}{in} \PY{n}{lyrics}\PY{p}{:}
                 \PY{n}{key} \PY{o}{=} \PY{l+s+s2}{\PYZdq{}}\PY{l+s+s2}{\PYZlt{}START\PYZgt{}}\PY{l+s+s2}{\PYZdq{}}
                 \PY{k}{for} \PY{n}{word} \PY{o+ow}{in} \PY{n}{lyric}\PY{o}{.}\PY{n}{split}\PY{p}{(}\PY{p}{)}\PY{p}{:}
                     \PY{n}{chain}\PY{p}{[}\PY{n}{key}\PY{p}{]}\PY{o}{.}\PY{n}{append}\PY{p}{(}\PY{n}{word}\PY{p}{)}
                     \PY{n}{key} \PY{o}{=} \PY{n}{word}
                     \PY{k}{if} \PY{n}{key} \PY{o+ow}{not} \PY{o+ow}{in} \PY{n}{chain}\PY{p}{:}
                         \PY{n}{chain}\PY{p}{[}\PY{n}{key}\PY{p}{]} \PY{o}{=} \PY{p}{[}\PY{p}{]}
                 \PY{n}{chain}\PY{p}{[}\PY{n}{key}\PY{p}{]}\PY{o}{.}\PY{n}{append}\PY{p}{(}\PY{l+s+s2}{\PYZdq{}}\PY{l+s+s2}{\PYZlt{}END\PYZgt{}}\PY{l+s+s2}{\PYZdq{}}\PY{p}{)}
             \PY{k}{return} \PY{n}{chain}
\end{Verbatim}


    \begin{Verbatim}[commandchars=\\\{\}]
{\color{incolor}In [{\color{incolor}16}]:} \PY{c+c1}{\PYZsh{} Load the pickled lyrics object that you created in Lab A.}
         \PY{k+kn}{import} \PY{n+nn}{pickle}
         \PY{n}{lyrics} \PY{o}{=} \PY{n}{pickle}\PY{o}{.}\PY{n}{load}\PY{p}{(}\PY{n+nb}{open}\PY{p}{(}\PY{l+s+s2}{\PYZdq{}}\PY{l+s+s2}{lyrics.pkl}\PY{l+s+s2}{\PYZdq{}}\PY{p}{,} \PY{l+s+s2}{\PYZdq{}}\PY{l+s+s2}{rb}\PY{l+s+s2}{\PYZdq{}}\PY{p}{)}\PY{p}{)}
         
         \PY{c+c1}{\PYZsh{} Call the function you wrote above.}
         \PY{n}{chain} \PY{o}{=} \PY{n}{train\PYZus{}markov\PYZus{}chain}\PY{p}{(}\PY{n}{lyrics}\PY{p}{)}
         
         \PY{c+c1}{\PYZsh{} What words tend to start a song (i.e., what words follow the \PYZlt{}START\PYZgt{} tag?)}
         \PY{n+nb}{print}\PY{p}{(}\PY{n}{chain}\PY{p}{[}\PY{l+s+s2}{\PYZdq{}}\PY{l+s+s2}{\PYZlt{}START\PYZgt{}}\PY{l+s+s2}{\PYZdq{}}\PY{p}{]}\PY{p}{)}
         
         \PY{c+c1}{\PYZsh{} What words tend to begin a line (i.e., what words follow the line break tag?)}
         \PY{n+nb}{print}\PY{p}{(}\PY{n}{chain}\PY{p}{[}\PY{l+s+s2}{\PYZdq{}}\PY{l+s+s2}{\PYZlt{}N\PYZgt{}}\PY{l+s+s2}{\PYZdq{}}\PY{p}{]}\PY{p}{[}\PY{p}{:}\PY{l+m+mi}{20}\PY{p}{]}\PY{p}{)}
\end{Verbatim}


    \begin{Verbatim}[commandchars=\\\{\}]
['Just', 'Waiting', 'I', 'I', 'Every', 'Baby', 'I', 'Do', 'Are', 'Yeah,', 'You', 'No', 'Easy', 'Who', 'I', "I've", "I've", "I've", "I've", 'Walking', 'You', 'You', 'Treat', 'Any', 'Thought', "It's", 'I', 'I', 'You', 'You', 'Jump', 'I', 'I', 'Meant', "It's", "I've", 'Ooh', 'Ooh']
['I', 'I', 'I', 'And', 'Everything', 'Think', 'Hoping', '<N>', '<N>', 'I', 'I', 'You', '<N>', '<N>', 'I', 'Cause', 'And', 'Cause', 'Cause', '<N>']

    \end{Verbatim}

    Now, let's generate new lyrics using the Markov chain you constructed
above. To do this, we'll begin at the
\texttt{"\textless{}START\textgreater{}"} state and randomly sample a
word from the list of words that follow
\texttt{"\textless{}START\textgreater{}"}. Then, at each step, we'll
randomly sample the next word from the list of words that followed each
current word. We will continue this process until we sample the
\texttt{"\textless{}END\textgreater{}"} state. This will give us the
complete lyrics of a randomly generated song!

You may find the \texttt{random.choice()} function helpful for this
question.

    \begin{Verbatim}[commandchars=\\\{\}]
{\color{incolor}In [{\color{incolor}17}]:} \PY{k+kn}{import} \PY{n+nn}{random}
         
         \PY{k}{def} \PY{n+nf}{generate\PYZus{}new\PYZus{}lyrics}\PY{p}{(}\PY{n}{chain}\PY{p}{)}\PY{p}{:}
             \PY{l+s+sd}{\PYZdq{}\PYZdq{}\PYZdq{}}
         \PY{l+s+sd}{    Args:}
         \PY{l+s+sd}{      \PYZhy{} chain: a dict representing the Markov chain,}
         \PY{l+s+sd}{               such as one generated by generate\PYZus{}new\PYZus{}lyrics()}
         \PY{l+s+sd}{    }
         \PY{l+s+sd}{    Returns:}
         \PY{l+s+sd}{      A string representing the randomly generated song.}
         \PY{l+s+sd}{    \PYZdq{}\PYZdq{}\PYZdq{}}
             
             \PY{c+c1}{\PYZsh{} a list for storing the generated words}
             \PY{n}{words} \PY{o}{=} \PY{p}{[}\PY{p}{]}
             \PY{n}{key} \PY{o}{=} \PY{l+s+s2}{\PYZdq{}}\PY{l+s+s2}{\PYZlt{}START\PYZgt{}}\PY{l+s+s2}{\PYZdq{}}
             \PY{k}{while} \PY{n}{key} \PY{o}{!=} \PY{l+s+s2}{\PYZdq{}}\PY{l+s+s2}{\PYZlt{}END\PYZgt{}}\PY{l+s+s2}{\PYZdq{}}\PY{p}{:}
                 \PY{n}{link} \PY{o}{=} \PY{n}{random}\PY{o}{.}\PY{n}{choice}\PY{p}{(}\PY{n}{chain}\PY{p}{[}\PY{n}{key}\PY{p}{]}\PY{p}{)}
                 \PY{n}{words}\PY{o}{.}\PY{n}{append}\PY{p}{(}\PY{n}{link}\PY{p}{)}
                 \PY{n}{key} \PY{o}{=} \PY{n}{link}
             
             \PY{c+c1}{\PYZsh{} join the words together into a string with line breaks}
             \PY{n}{lyrics} \PY{o}{=} \PY{l+s+s2}{\PYZdq{}}\PY{l+s+s2}{ }\PY{l+s+s2}{\PYZdq{}}\PY{o}{.}\PY{n}{join}\PY{p}{(}\PY{n}{words}\PY{p}{[}\PY{p}{:}\PY{o}{\PYZhy{}}\PY{l+m+mi}{1}\PY{p}{]}\PY{p}{)}
             \PY{k}{return} \PY{l+s+s2}{\PYZdq{}}\PY{l+s+se}{\PYZbs{}n}\PY{l+s+s2}{\PYZdq{}}\PY{o}{.}\PY{n}{join}\PY{p}{(}\PY{n}{lyrics}\PY{o}{.}\PY{n}{split}\PY{p}{(}\PY{l+s+s2}{\PYZdq{}}\PY{l+s+s2}{\PYZlt{}N\PYZgt{}}\PY{l+s+s2}{\PYZdq{}}\PY{p}{)}\PY{p}{)}
\end{Verbatim}


    \begin{Verbatim}[commandchars=\\\{\}]
{\color{incolor}In [{\color{incolor}20}]:} \PY{n+nb}{print}\PY{p}{(}\PY{n}{generate\PYZus{}new\PYZus{}lyrics}\PY{p}{(}\PY{n}{chain}\PY{p}{)}\PY{p}{)}
\end{Verbatim}


    \begin{Verbatim}[commandchars=\\\{\}]
Walking down 
 Say it 
 I'm gonna rush 
 My side I haven't ruined all battling fear the blinding light 
 I've gotta get to be? 
 
 If I stare at the ways that castle's got a way 
 You will work it all on my heart alone 
 And I'm looking at the thing through hell and you how little I tried so hard to make you make a sound? 
 Take all of something great 
 
 Yeah I feel like you're brave enough 
 I'm looking out for sure 
 
 
 
 So don't let someone find you 
 Ya and far past you're sorry 
 You're the table 
 
 
 
 No denying 
 Are you know 
 I'm looking down 
 You're the night my heart alone 
 I'm gonna love it right now I wanna feel like trying 
 I'll show me where the thing you got room for me that 
 But you 
 
 I'm gonna love 
 I'm running my body said, no other sides? 
 See me high 
 I'm never say 
 And I won't make those things I either way 
 What do 
 We'll find that I'm my mind 
 And don't change 
 
 
 You can really hold on 
 
 I'm playing the love 
 Memories are 
 She's an easy lover 
 I need 
 
 Something to get out for me 
 Ya I'm crazy 
 Now I'm driving 
 I've given you got me 
 
 I've been trying 
 Why would stay the things you when it's already love it makes me 
 You are you? I don't let someone find you might be 
 I've been running my phone and I know the times that I'm not up to think something to show you take with nothing to giving up alone

    \end{Verbatim}

    \hypertarget{question-3.-bigram-markov-chain-model}{%
\section{Question 3. Bigram Markov Chain
Model}\label{question-3.-bigram-markov-chain-model}}

Now you'll build a more complex Markov chain that uses the last
\emph{two} words (or bigram) to predict the next word. Now your dict
\texttt{chain} should map a \emph{tuple} of words to a list of words
that appear after it.

As before, you should also include tags that indicate the beginning and
end of a song, as well as line breaks. That is, a tuple might contain
tags like \texttt{"\textless{}START\textgreater{}"},
\texttt{"\textless{}END\textgreater{}"}, and
\texttt{"\textless{}N\textgreater{}"}, in addition to regular words. So
for example, for
\href{https://www.azlyrics.com/lyrics/stevemillerband/thejoker.html}{``The
Joker''}, you would add the following to your chain:

\begin{verbatim}
chain = {
    (None, "<START>"): ["Some", ...],
    ("<START>", "Some"): ["people", ...],
    ("Some", "people"): ["call", ...],
    ("people", "call"): ["me", ...],
    ("call", "me"): ["the", ...],
    ("me", "the"): ["space", ...],
    ("the", "space"): ["cowboy,", ...],
    ("space", "cowboy,"): ["yeah", ...],
    ("cowboy,", "yeah"): ["<N>", ...],
    ("yeah", "<N>"): ["Some", ...],
    ("time", "<N>"): ["Come"],
    ...,
    ("<N>", "Come"): ["on", ...],
    ("Come", "on"): ["baby", ...],
    ("on", "baby"): ["and", ...],
    ("baby", "and"): ["I'll", ...],
    ("and", "I'll"): ["show", ...],
    ("I'll", "show"): ["you", ...],
    ("show", "you"): ["a", ...],
    ("you", "a"): ["good", ...],
    ("a", "good"): ["time", ...],
    ("good", "time"): ["<END>", ...],
}
\end{verbatim}

    \begin{Verbatim}[commandchars=\\\{\}]
{\color{incolor}In [{\color{incolor}10}]:} \PY{k}{def} \PY{n+nf}{train\PYZus{}markov\PYZus{}chain}\PY{p}{(}\PY{n}{lyrics}\PY{p}{)}\PY{p}{:}
             \PY{l+s+sd}{\PYZdq{}\PYZdq{}\PYZdq{}}
         \PY{l+s+sd}{    Args:}
         \PY{l+s+sd}{      \PYZhy{} lyrics: a list of strings, where each string represents}
         \PY{l+s+sd}{                the lyrics of one song by an artist.}
         \PY{l+s+sd}{    }
         \PY{l+s+sd}{    Returns:}
         \PY{l+s+sd}{      A dict that maps a tuple of 2 words (\PYZdq{}bigram\PYZdq{}) to a list of}
         \PY{l+s+sd}{      words that follow that bigram, representing the Markov}
         \PY{l+s+sd}{      chain trained on the lyrics.}
         \PY{l+s+sd}{    \PYZdq{}\PYZdq{}\PYZdq{}}
             \PY{n}{chain} \PY{o}{=} \PY{p}{\PYZob{}}\PY{p}{(}\PY{k+kc}{None}\PY{p}{,} \PY{l+s+s2}{\PYZdq{}}\PY{l+s+s2}{\PYZlt{}START\PYZgt{}}\PY{l+s+s2}{\PYZdq{}}\PY{p}{)}\PY{p}{:} \PY{p}{[}\PY{p}{]}\PY{p}{\PYZcb{}}
             \PY{k}{for} \PY{n}{lyric} \PY{o+ow}{in} \PY{n}{lyrics}\PY{p}{:}
                 \PY{n}{key} \PY{o}{=} \PY{p}{(}\PY{k+kc}{None}\PY{p}{,} \PY{l+s+s2}{\PYZdq{}}\PY{l+s+s2}{\PYZlt{}START\PYZgt{}}\PY{l+s+s2}{\PYZdq{}}\PY{p}{)}
                 \PY{k}{for} \PY{n}{word} \PY{o+ow}{in} \PY{n}{lyric}\PY{o}{.}\PY{n}{split}\PY{p}{(}\PY{p}{)}\PY{p}{:}
                     \PY{n}{chain}\PY{p}{[}\PY{n}{key}\PY{p}{]}\PY{o}{.}\PY{n}{append}\PY{p}{(}\PY{n}{word}\PY{p}{)}
                     \PY{n}{key} \PY{o}{=} \PY{p}{(}\PY{n}{key}\PY{p}{[}\PY{l+m+mi}{1}\PY{p}{]}\PY{p}{,} \PY{n}{word}\PY{p}{)}
                     \PY{k}{if} \PY{n}{key} \PY{o+ow}{not} \PY{o+ow}{in} \PY{n}{chain}\PY{p}{:}
                         \PY{n}{chain}\PY{p}{[}\PY{n}{key}\PY{p}{]} \PY{o}{=} \PY{p}{[}\PY{p}{]}
                 \PY{n}{chain}\PY{p}{[}\PY{n}{key}\PY{p}{]}\PY{o}{.}\PY{n}{append}\PY{p}{(}\PY{l+s+s2}{\PYZdq{}}\PY{l+s+s2}{\PYZlt{}END\PYZgt{}}\PY{l+s+s2}{\PYZdq{}}\PY{p}{)}
             \PY{k}{return} \PY{n}{chain}
\end{Verbatim}


    \begin{Verbatim}[commandchars=\\\{\}]
{\color{incolor}In [{\color{incolor}11}]:} \PY{c+c1}{\PYZsh{} Load the pickled lyrics object that you created in Lab A.}
         \PY{k+kn}{import} \PY{n+nn}{pickle}
         \PY{n}{lyrics} \PY{o}{=} \PY{n}{pickle}\PY{o}{.}\PY{n}{load}\PY{p}{(}\PY{n+nb}{open}\PY{p}{(}\PY{l+s+s2}{\PYZdq{}}\PY{l+s+s2}{lyrics.pkl}\PY{l+s+s2}{\PYZdq{}}\PY{p}{,} \PY{l+s+s2}{\PYZdq{}}\PY{l+s+s2}{rb}\PY{l+s+s2}{\PYZdq{}}\PY{p}{)}\PY{p}{)}
         
         \PY{c+c1}{\PYZsh{} Call the function you wrote above.}
         \PY{n}{chain} \PY{o}{=} \PY{n}{train\PYZus{}markov\PYZus{}chain}\PY{p}{(}\PY{n}{lyrics}\PY{p}{)}
         
         \PY{c+c1}{\PYZsh{} What words tend to start a song (i.e., what words follow the \PYZlt{}START\PYZgt{} tag?)}
         \PY{n+nb}{print}\PY{p}{(}\PY{n}{chain}\PY{p}{[}\PY{p}{(}\PY{k+kc}{None}\PY{p}{,} \PY{l+s+s2}{\PYZdq{}}\PY{l+s+s2}{\PYZlt{}START\PYZgt{}}\PY{l+s+s2}{\PYZdq{}}\PY{p}{)}\PY{p}{]}\PY{p}{)}
         
         \PY{n+nb}{print}\PY{p}{(}\PY{n}{chain}\PY{p}{[}\PY{p}{(}\PY{l+s+s2}{\PYZdq{}}\PY{l+s+s2}{\PYZlt{}START\PYZgt{}}\PY{l+s+s2}{\PYZdq{}}\PY{p}{,} \PY{l+s+s2}{\PYZdq{}}\PY{l+s+s2}{Just}\PY{l+s+s2}{\PYZdq{}}\PY{p}{)}\PY{p}{]}\PY{p}{)}
\end{Verbatim}


    \begin{Verbatim}[commandchars=\\\{\}]
['Just', 'Waiting', 'I', 'I', 'Every', 'Baby', 'I', 'Do', 'Are', 'Yeah,', 'You', 'No', 'Easy', 'Who', 'I', "I've", "I've", "I've", "I've", 'Walking', 'You', 'You', 'Treat', 'Any', 'Thought', "It's", 'I', 'I', 'You', 'You', 'Jump', 'I', 'I', 'Meant', "It's", "I've", 'Ooh', 'Ooh']
['forget']

    \end{Verbatim}

    Now, let's generate new lyrics using the Markov chain you constructed
above. To do this, we'll begin at the
\texttt{(None,\ "\textless{}START\textgreater{}")} state and randomly
sample a word from the list of words that follow this bigram. Then, at
each step, we'll randomly sample the next word from the list of words
that followed the current bigram (i.e., the last two words). We will
continue this process until we sample the
\texttt{"\textless{}END\textgreater{}"} state. This will give us the
complete lyrics of a randomly generated song!

    \begin{Verbatim}[commandchars=\\\{\}]
{\color{incolor}In [{\color{incolor}12}]:} \PY{k+kn}{import} \PY{n+nn}{random}
         
         \PY{k}{def} \PY{n+nf}{generate\PYZus{}new\PYZus{}lyrics}\PY{p}{(}\PY{n}{chain}\PY{p}{)}\PY{p}{:}
             \PY{l+s+sd}{\PYZdq{}\PYZdq{}\PYZdq{}}
         \PY{l+s+sd}{    Args:}
         \PY{l+s+sd}{      \PYZhy{} chain: a dict representing the Markov chain,}
         \PY{l+s+sd}{               such as one generated by generate\PYZus{}new\PYZus{}lyrics()}
         \PY{l+s+sd}{    }
         \PY{l+s+sd}{    Returns:}
         \PY{l+s+sd}{      A string representing the randomly generated song.}
         \PY{l+s+sd}{    \PYZdq{}\PYZdq{}\PYZdq{}}
             
             \PY{c+c1}{\PYZsh{} a list for storing the generated words}
             \PY{n}{words} \PY{o}{=} \PY{p}{[}\PY{p}{]}
             
             \PY{c+c1}{\PYZsh{} YOUR CODE HERE}
             \PY{n}{key} \PY{o}{=} \PY{p}{(}\PY{k+kc}{None}\PY{p}{,} \PY{l+s+s2}{\PYZdq{}}\PY{l+s+s2}{\PYZlt{}START\PYZgt{}}\PY{l+s+s2}{\PYZdq{}}\PY{p}{)}
             \PY{k}{while} \PY{n}{key}\PY{p}{[}\PY{l+m+mi}{1}\PY{p}{]} \PY{o}{!=} \PY{l+s+s2}{\PYZdq{}}\PY{l+s+s2}{\PYZlt{}END\PYZgt{}}\PY{l+s+s2}{\PYZdq{}}\PY{p}{:}
                 \PY{n}{link} \PY{o}{=} \PY{n}{random}\PY{o}{.}\PY{n}{choice}\PY{p}{(}\PY{n}{chain}\PY{p}{[}\PY{n}{key}\PY{p}{]}\PY{p}{)}
                 \PY{n}{words}\PY{o}{.}\PY{n}{append}\PY{p}{(}\PY{n}{link}\PY{p}{)}
                 \PY{n}{key} \PY{o}{=} \PY{p}{(}\PY{n}{key}\PY{p}{[}\PY{l+m+mi}{1}\PY{p}{]}\PY{p}{,} \PY{n}{link}\PY{p}{)}
             
             \PY{c+c1}{\PYZsh{} join the words together into a string with line breaks}
             \PY{n}{lyrics} \PY{o}{=} \PY{l+s+s2}{\PYZdq{}}\PY{l+s+s2}{ }\PY{l+s+s2}{\PYZdq{}}\PY{o}{.}\PY{n}{join}\PY{p}{(}\PY{n}{words}\PY{p}{[}\PY{p}{:}\PY{o}{\PYZhy{}}\PY{l+m+mi}{1}\PY{p}{]}\PY{p}{)}
             \PY{k}{return} \PY{l+s+s2}{\PYZdq{}}\PY{l+s+se}{\PYZbs{}n}\PY{l+s+s2}{\PYZdq{}}\PY{o}{.}\PY{n}{join}\PY{p}{(}\PY{n}{lyrics}\PY{o}{.}\PY{n}{split}\PY{p}{(}\PY{l+s+s2}{\PYZdq{}}\PY{l+s+s2}{\PYZlt{}N\PYZgt{}}\PY{l+s+s2}{\PYZdq{}}\PY{p}{)}\PY{p}{)}
\end{Verbatim}


    \begin{Verbatim}[commandchars=\\\{\}]
{\color{incolor}In [{\color{incolor}14}]:} \PY{n+nb}{print}\PY{p}{(}\PY{n}{generate\PYZus{}new\PYZus{}lyrics}\PY{p}{(}\PY{n}{chain}\PY{p}{)}\PY{p}{)}
\end{Verbatim}


    \begin{Verbatim}[commandchars=\\\{\}]
Jump out of the heights 
 With sting with bite 
 I'm gonna tell you how I feel like, feel like, feel like I'll never change 
 I feel like, feel like trying 
 
 I saw you I am dying 
 To say that it's me? 
 Forget about the past you're holding 
 Move on to 
 Lie on the line 
 If you're brave enough to love it if we try 
 But you got your things to hold her 
 You'd better forget it 
 Will you come here to see it's not my time 
 Love you 
 How will I let you go? 
 You were there for me, and I was doing 
 Turn salt to pearl 
 True no duke no earl 
 But from the other way 
 I know that I'm crazy 
 Cause it's already love 
 
 I won't bruise you either way 
 I've been running my whole life 
 Say it all 
 Say it all 
 I've come to get you I'm all I find 
 Every time I ever fought it 
 Oh ooh oh oh ooh 
 It's almost too much to handle

    \end{Verbatim}

    \hypertarget{analysis}{%
\section{Analysis}\label{analysis}}

Compare the quality of the lyrics generated by the unigram model (in Lab
B) and the bigram model (in Lab C). Which model seems to generate more
reasonable lyrics? Can you explain why? What do you see as the
advantages and disadvantages of each model?

    It appears that the bi-gram markov chain generates more cohesive (both
gramatically and lyrically) song lyrics. This is because (at least for
the grammar part) it is sampling a bi-gram from naturally occuring two
word sequences. Therefore the words ``I've been running my'' (``I've
been'', ``been running'', ``running my'') appear naturally frequently
together in the song lyrics versus ``I've given you got'' (``I've'',
``given'', ``you'', ``got''). To be clearer, the bigrams are more
probabilistically prone to be more grammatically correct than the
unigrams

    \hypertarget{submission-instructions}{%
\section{Submission Instructions}\label{submission-instructions}}

Once you are finished, follow these steps:

\begin{enumerate}
\def\labelenumi{\arabic{enumi}.}
\tightlist
\item
  Restart the kernel and re-run this notebook from beginning to end by
  going to
  \texttt{Kernel\ \textgreater{}\ Restart\ Kernel\ and\ Run\ All\ Cells}.
\item
  If this process stops halfway through, that means there was an error.
  Correct the error and repeat Step 1 until the notebook runs from
  beginning to end.
\item
  Double check that there is a number next to each code cell and that
  these numbers are in order.
\end{enumerate}

Then, submit your lab as follows:

\begin{enumerate}
\def\labelenumi{\arabic{enumi}.}
\tightlist
\item
  Go to
  \texttt{File\ \textgreater{}\ Export\ Notebook\ As\ \textgreater{}\ PDF}.
\item
  Double check that the entire notebook, from beginning to end, is in
  this PDF file. (If the notebook is cut off, try first exporting the
  notebook to HTML and printing to PDF.)
\item
  Upload the PDF
  \href{https://polylearn.calpoly.edu/AY_2018-2019/mod/assign/view.php?id=349486}{to
  PolyLearn}.
\end{enumerate}


    % Add a bibliography block to the postdoc
    
    
    
    \end{document}
